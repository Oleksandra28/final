\cleardoublepage

%---------------------------------------------------------------------------------------------------------------------
\chapter{Data Description}
\label{datadescription}

In this chapter, we first present the original dataset in Section \ref{originaldataset}, and then describe the operations performed on the data (i.e. preprocessing) in Section \ref{datapreprocessing}. 

%---------------------------------------------------------------------------------------------------------------------
\section{Original Dataset}
\label{originaldataset}

In this work, we use the dataset CrisisLexT6 ~\citep{data}. The dataset consists of 60,000 tweets\footnote{A tweet is a short text (up to 140 characters) that users of Twitter can post on Twitter.com} posted during 6 crisis events in 2012 and 2013. The 60,000 tweets (10,000 in each disaster) have been labeled by crowdsourcing workers according to relatedness (as \textit{on-topic} or \textit{off-topic}). The \textit{on-topic} tweets are labeled as \textit{1}, and the \textit{off-topic} tweets are labeled as \textit{0}. The amount of tweets per class for each disaster is presented in Table \ref{originaldatasettable}.

%==================Table 1 begin
\begin{table}[ht]
    \begin{center}
    \caption{Original Dataset}
    \begin{tabular}[c]{|c|c|c|c|}
        \hline
        Crisis & On-topic & Off-topic & Total \\
        \hline
        2012 Sandy Hurricane & 6138 & 3870 & 10008 \\
        2013 Queensland Floods & 5414 & 4619 & 10033 \\ 
        2013 Boston Bombings & 5648 & 4364 & 10012 \\ 
        2013 West Texas Explosion & 5246 & 4760 & 10006 \\
        2013 Oklahoma Tornado & 4827 & 5165 & 9992 \\
        2013 Alberta Floods & 5189 & 4842 & 10031 \\
        \hline
    \end{tabular}
    \label{originaldatasettable}
   \end{center}
\end{table}
%==================Table 1 end


%---------------------------------------------------------------------------------------------------------------------
\section{Data Preprocessing}
\label{datapreprocessing}

The tweets are preprocessed before they are used in training, domain adaptation and testing stages.
The following cleaning steps have been taken ~\citep{twitterda}:

\begin{itemize}
  \item non-printable, ASCII characters are removed, as they are generally regarded as noise rather than useful information.
  \item printable HTML entities are converted into their corresponding ASCII equivalents
  \item URLs, email addresses, and usernames are replaced with a URL/email/username placeholder for each type of entity, respectively, under the assumption that those features could be predictive 
  \item numbers, punctuation signs and hashtags are kept under the assumption that numbers could be indicative of an address, while punctuation/emoticons and hashtags could be indicative of emotions 
  \item RT (i.e., retweet) are removed under the assumptions that such features are not informative for our classification tasks
  \item duplicate tweets and empty tweets (that have no characters left after the cleaning) are removed from the data sets
\end{itemize}

The amount of tweets per class for each disaster has reduced and is presented in Table \ref{aftercleaningdatasettable}.

%==================Table 1 begin
\begin{table}[ht]
    \begin{center}
    \caption{Dataset after preprocessing}
    \begin{tabular}[c]{|c|c|c|c|}
        \hline
        Crisis & On-topic & Off-topic & Total \\
        \hline
        2012 Sandy Hurricane & 5261 & 3752 & 9013 \\
        2013 Queensland Floods & 3236 & 4550 & 7786 \\ 
        2013 Boston Bombings & 4441 & 4309 & 8750 \\ 
        2013 West Texas Explosion & 4123 & 4733 & 8856 \\
        2013 Oklahoma Tornado & 3209 & 5049 & 8258 \\
        2013 Alberta Floods & 3497 & 4714 & 8211 \\
        \hline
    \end{tabular}
    \label{aftercleaningdatasettable}
   \end{center}
\end{table}
%==================Table 1 end

After preprocessing, the source tweets are expressed via target features i.e. via words that occur in the target tweets. The bag-of-words ~\citep{tom} representation is used to represent tweets as vectors of features. A sample bag-of-words tweet represenatation is presented in Table \ref{sampletweets}.


%==================Table 1 begin
\begin{table}[ht]
    \begin{center}
    \caption{Example of bag-of-words representation}
    \begin{tabular}[c]{|c|c|c|c|c|c|c|c|c|c|c|c|}
        \hline
        Sample Tweet & life & is & real & now & New & York & has & been & real & travelling & to \\ 
        \hline
        Life is real now & 1 & 1 & 1 & 1 & 0 & 0 & 0 & 0 & 0 & 0 & 0 \\
        Travelling to New York now & 0 & 0 & 0 & 1 & 1 & 1 & 0 & 0 & 0 & 1 & 1 \\
        New York has been real & 0 & 0 & 1 & 0 & 1 & 1 & 1 & 1 & 1 & 0 & 0 \\
        \hline
    \end{tabular}
    \label{sampletweets}
   \end{center}
\end{table}
%==================Table 1 end



