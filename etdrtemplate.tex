% +--------------------------------------------------------------------+
% | LaTeX Template for K-State Electronic Theses, Dissertations,
% | and Reports
% |
% | Some guidelines for using the template are shown in comments.  Read
% | these comments carefully, as they describe changes you will need to
% | make to the template in order to meet Graduate School requirements.            |
% |
% | Additional information on using the template are contained in these
% | files, which are included when you download the template:
% |
% | ReadMe.pdf - A general overview of using the template
% |
% | BibTeX Guide.pdf - Detailed guidelines on using BibTeX to create
% | your bibliography and manage your citations.
% |
% | natbib.pdf - Gives detailed information on using the natbib package
% | and formatting citations
% +--------------------------------------------------------------------+

% +--------------------------------------------------------------------+
% | The template is designed to be used with PDFLaTeX. Process this
% | file (etdrtemplate.tex) with PDFLaTeX in order to produce a PDF
% | version of your ETDR.  If you are using BibTex to manage your
% | refrences, you will need to process your file four times:
% | 1. Run PDFLaTeX
% | 2. Run BibTex
% | 3. Run PDFLaTeX
% | 4. Run PDFLaTeX
% |
% | Some LaTeX editors do not explicitly list PDFLaTeX as an option, but
% | do use PDFLaTeX to produce a PDF file directly from your .tex files.
% | See the ReadMe file for details.
% +--------------------------------------------------------------------+
% |
% +--------------------------------------------------------------------+
% |
% | As required by the Graduate School, The template is configured to
% | contain the following sections in the order shown.

% | Title page
% | Copyright page
% | Abstract
% | Table of contents
% | List of figures
% | List of tables
% | Acknowledgements (Optional)
% | Dedication (Optional)
% | Preface (Optional)
% | Individual Chapters
% | References or Bibliography
% | Appendices (as needed)
% |
% | Details on removing optional sections are given in the comments below.
% |
% +--------------------------------------------------------------------+

% +--------------------------------------------------------------------+
% | The LaTex command \documentclass selects a particular class to
% | associate with the document.  Within this command, 12pt is
% | specified for the font size.  You can change this to 11 pt, if
% | desired.
% +--------------------------------------------------------------------+

\documentclass[final,letterpaper,12pt,oneside]{class_diss}

% +--------------------------------------------------------------------+
% | Here are added external packages that will be used throughout
% | the document.  You can add other packages as needed.
% +--------------------------------------------------------------------+

\usepackage{graphicx} % Extended graphics package.
\usepackage{amsmath} % American Mathematics Society standards
\usepackage{amsxtra} % Additional math symbols
\usepackage{amssymb} % Additional math symbols
\usepackage{amsthm} % Additional math symbols
\usepackage{latexsym} % Additional math symbols
\usepackage{setspace} % Controls line spacing
\usepackage[margin=1in]{geometry} % Sets page margins to 1 inch on all sides
\usepackage[titles]{tocloft} % Adds leader dots to all entries in the table of contents

% +--------------------------------------------------------------------+
% |
% | Citation and Bibliography Style
% |
% | The following commands determine the citation and bibliography style.  The
% | template uses BibTeX for formatting the bibliography.  See "BibTeX Guide.pdf"
% | for details on formatting citations and references.  The template is set
% | to use a generic, superscript style, but it can be easily modified
% | to use author-year styles.

%\bibliographystyle{unsrtnat}
\bibliographystyle{plainnat}
% | If you want to use an author-year citation style, change "unsrtnat" to
% | "plainnat" in the line above.  You can also use other styles supported
% | by LaTeX, e.g., acm, ieeetr, siam, etc.  Additional styles
% | are in the \styles folder and can be invoked like this:
% | \bibliographystyle{styles/apsrev}.  If the style you use is based
% | on an author-year citation style, you will need to make changes
% | in the usepackage and \setcitestyle statements below

\usepackage[super,sort&compress]{natbib}
%\bibpunct{[}{]}{;}{a}{,}{,}
\bibpunct{[}{]}{a}{,}{,}
% | If you want to use an author-year citation style, change "super" to
% | "authoryear" in the line above.

\usepackage{pgfplots}

% \setcitestyle{super}
\setcitestyle{authoryear}
% | if you want to use an author-year citation style, change "super" to
% | "authoryear" in the line above.

% +---------------------------------------------------------------------+
% | The hyperref package enables cross-references.
% +---------------------------------------------------------------------+

\usepackage[pdftex, plainpages=false, pdfpagelabels]{hyperref}

\hypersetup{
    linktocpage=true,
    colorlinks=true,
    bookmarks=true,
    citecolor=blue,
    urlcolor=blue,
    linkcolor=blue,
    citebordercolor={1 0 0},
    urlbordercolor={1 0 0},
    linkbordercolor={.7 .8 .8},
    breaklinks=true,
    pdfpagelabels=true,
    }

% +--------------------------------------------------------------------+
% | The document begins here.
% +--------------------------------------------------------------------+

\doublespacing
\begin{document}

% +--------------------------------------------------------------------+
% | Title Page -- Required for both Doctoral and Masters Students
% +--------------------------------------------------------------------+

% +--------------------------------------------------------------------+
% | Title Page
% +--------------------------------------------------------------------+

\newpage

% +--------------------------------------------------------------------+
% | This page should not contain a page number.  We use the
% | \thispagestyle[empty] command below to suppress page numbers
% | and other style elements.
% +--------------------------------------------------------------------+

\thispagestyle{empty}

% +--------------------------------------------------------------------+
% | The Title page begins here.
% +--------------------------------------------------------------------+

\begin{center}

   \vspace{1cm}

% +--------------------------------------------------------------------+
% | On the line below, replace "ENTER YOUR TITLE" with the title of
% | your ETDR.  Use all CAPITAL LETTERS.
% +--------------------------------------------------------------------+

   \large Domain Adaptation for Disaster-related Twitter Data \\

   \vspace{0.3cm}

   by\\

   \vspace{0.3cm}

% +--------------------------------------------------------------------+
% | On the line below, replace "ENTER YOUR NAME" with your name.  Use
% | mixed case, for example, Laura Bush.
% +--------------------------------------------------------------------+

   \large Oleksandra Sopova\\

   \vspace{0.3cm}

% +--------------------------------------------------------------------+
% | On the line below, replace "Enter Your Previous Degrees"
% | with your previous degrees in mixed case. Include the abbreviation
% | for the degree, the name of the university, and the year separated
% | by commas. For example:
% |
% |    B.A., University of Illinois, 2000
% |
% | If desired, it is acceptable to include a city or country with
% | the university name. For example:
% |
% |    B.S., Jillian University, China, 2002
% |
% | Each degree should appear on a separate line.  Use the \\
% | command to create a line break.
% +--------------------------------------------------------------------+

   B.S., National University of “Kyiv-Mogyla Academy”, 2015 \\

   \vspace{0.35cm}
   \rule{2in}{0.5pt}\\
   \vspace{0.65cm}

   {\large A REPORT}\\

   \vspace{0.3cm}
   \begin{singlespace}
   submitted in partial fulfillment of the\\
   requirements for the degree\\
   \end{singlespace}

   \vspace{0.3cm}

% +--------------------------------------------------------------------+
% | On the line below, replace "ENTER YOUR DEGREE NAME" with the name
% | of your earned degree in ALL CAPITAL LETTERS.
% +--------------------------------------------------------------------+

   {\large MASTER OF SCIENCE}\\
   \vspace{0.3cm}

% +--------------------------------------------------------------------+
% | On the two lines below, replace "Enter Your Department Name" and
% | "Enter Your College Name" with the name of your department and the
% | name of the college in mixed case.  For example:
% |
% |     Biochemistry Department
% |     College of Arts and Sciences
% +--------------------------------------------------------------------+

   \begin{singlespace}
   Department of Computer Science \\
   College of Engineering\\
   \end{singlespace}

   \vspace{0.3cm}

   \begin{singlespace}
   {\large KANSAS STATE UNIVERSITY}\\
   Manhattan, Kansas\\
   \end{singlespace}

% +--------------------------------------------------------------------+
% | On the line below, replace "Graduation Year" with the four-digit
% | year of your graduation.  For example:
% |
% |     2016
% +--------------------------------------------------------------------+

   2017\\
   \vspace{0.3cm}

    \end{center}

    \begin{flushright}
    Approved by:\\
    \vspace{0.3cm}
    \begin{singlespace}
    Major Professor


% +--------------------------------------------------------------------+
% | On the line below, replace "Enter Your Major Professor's Name"
% | with  the name of your major professor in mixed case.  Use the
% | format Firstname Lastname.  For example:
% |
% |     Lori Goetsch
% |
% +--------------------------------------------------------------------+

    Doina Caragea\\
    \end{singlespace}
    \end{flushright}

% +--------------------------------------------------------------------+
% | If you have co-major professors, comment out the lines above from
% | \begin{flushright} through \end{flushright} and uncomment the
% | lines below.  Enter your co-major professors' names where indicated.
% +--------------------------------------------------------------------+

%\begin{flushright}
%   Approved by:\\
%  \vspace{ 0.3cm}
%   \begin{singlespace}
%   Co-Major Professor\\
%   Enter Your Co-Major Professor's Name\\
%   \vspace{.25cm}
%   Co-Major Professor\\
%   Enter Your Co-Major Professor's Name\\
%   \end{singlespace}
%\end{flushright}


% +--------------------------------------------------------------------+
% | Copyright Page -- Required for both Doctoral and Masters Students
% +--------------------------------------------------------------------+

% +--------------------------------------------------------------------+
% | Copyright Page
% +--------------------------------------------------------------------+

\newpage

\thispagestyle{empty}

\vspace*{0.9cm}

\begin{center}

{\bf \Huge Copyright}

\vspace{1cm}

% +--------------------------------------------------------------------+
% | On the line below, replace "Enter Your Name" with your name
% | Use the same form of your name as it appears on your title page.
% | Use mixed case, for example, Barack Obama.
% +--------------------------------------------------------------------+

   \Large \textcopyright  Oleksandra Sopova 2017.\\

   \vspace{0.5cm}

   \vspace{0.5cm}

\end{center}


% +--------------------------------------------------------------------+
% |  Abstract -- Required for both Doctoral and Masters Students
% +--------------------------------------------------------------------+

\begin{abstract}

% +--------------------------------------------------------------------+
% | For masters theses or reports, uncomment the commands on the next
% | two lines (\setcounter and \pdfbookmark)
% +--------------------------------------------------------------------+

   \setcounter{page}{-1}
   \pdfbookmark[0]{Abstract}{PDFAbstractPage}

% +--------------------------------------------------------------------+
% | Abstract Page
% +--------------------------------------------------------------------+

\pagestyle{empty}
%\vspace{1cm}
\setlength{\baselineskip}{0.8cm}

%\indent

% +--------------------------------------------------------------------+
% | Enter the text of your abstract below, maximum of 500 words.
% +--------------------------------------------------------------------+

Machine learning is the subfield of Artificial intelligence that gives computers the ability to learn without being explicitly programmed, as it was defined by Arthur Samuel - the American pioneer in the field of computer gaming and artificial intelligence who was born in Emporia, Kansas. 

Supervised Machine Learning is focused on building predictive models given labeled training data. Data may come from a variety of sources, for instance, social media networks. 

In our research, we use Twitter data, specifically, user-generated tweets about disasters such as floods, hurricanes, terrorist attacks, etc., to build classifiers that could help disaster management teams identify useful information. 

A supervised classifier trained on data (\textit{training data}) from a particular domain (i.e. disaster) is expected to give accurate predictions on unseen data (\textit{testing data}) from the same domain, assuming that the training and test data have similar characteristics. Labeled data is not easily available for a current \textit{target} disaster. 

However, labeled data from a prior \textit{source} disaster is presumably available, and can be used to learn a supervised classifier for the target disaster. 

Unfortunately, the source disaster data and the target disaster data may not share the same characteristics, and the classifier learned from the source may not perform well on the target. Domain adaptation techniques, which use unlabeled target data in addition to labeled source data, can be used to address this problem. 

We study single-source and multi-source domain adaptation techniques, using Naïve Bayes classifier. 

Experimental results on Twitter datasets corresponding to six disasters show that domain adaptation techniques improve the overall performance as compared to basic supervised learning classifiers. 

Domain adaptation is crucial for many machine learning applications, as it enables the use of unlabeled data in domains where labeled data is not available.


\vfill
\end{abstract}

% +--------------------------------------------------------------------+
% | The following commands start a new page and set the page numbering
% | to lowercase roman numerals.
% +--------------------------------------------------------------------+

\newpage
\pagenumbering{roman}

% +--------------------------------------------------------------------+
% |
% | *********************** IMPORTANT ******************************
% |
% | In the \setcounter command below, set the number to represent the
% | page number of the table of contents page.  For example, if the
% | table of contents page is the 6th page of your document, enter 6
% | in the brackets.  This number may vary, depending on the length of
% | your abstract.
% |
% | Numbers do not appear on the title and abstract pages, but they
% | are included in the page count.  The table of contents page is the
% | first page on which page numbers are displayed.
% +--------------------------------------------------------------------+

\setcounter{page}{4}

% +--------------------------------------------------------------------+
% | The following command creates a bookmark for the table of contents
% | in the final PDF document.
% +--------------------------------------------------------------------+

\pdfbookmark[0]{\contentsname}{contents}

% +--------------------------------------------------------=-----------+
% | The following command adds dot leaders for all entries in the
% | table of contents.
% +--------------------------------------------------------------------+

\renewcommand{\cftchapleader}{\cftdotfill{\cftdotsep}}

% +--------------------------------------------------------------------+
% | The following commands makes all entries and page numbers in the
% | table of contents appear in normal weight font (not bold).
% +--------------------------------------------------------------------+

\renewcommand{\cftchapfont}{\mdseries}
\renewcommand{\cftchappagefont}{\mdseries}

% +--------------------------------------------------------------------+
% | These commands add the table of contents, list of figures, and
% | list of tables.
% +--------------------------------------------------------------------+

\tableofcontents
\listoffigures
\listoftables

% +--------------------------------------------------------------------+
% | Acknowledgements Page
% |
% | If you choose not to have an Acknowledgements page, comment out
% | or delete the following 3 lines.
% +--------------------------------------------------------------------+

\phantomsection
\addcontentsline{toc}{chapter}{Acknowledgements}
% +--------------------------------------------------------------------+
% | Acknowledgements Page (Optional)
% +--------------------------------------------------------------------+

\newpage
\vspace*{0.9cm}
\begin{center}
{\bf \Huge Acknowledgments}
\end{center}

\setlength{\baselineskip}{0.8cm}

%\pdfbookmark[0]{Acknowledgements}{PDF_Acknowledgements}

% +--------------------------------------------------------------------+
% | Enter text for your acknowledgements in the space below this box.
% |                                                                    
% +--------------------------------------------------------------------+


I would like to thank my advisor Dr. Doina Caragea for her excellent advice, guidance, valueable comments, suggestions and inspiration throughout my Masters program, and for providing the freedom in doing research and exploring various ideas.  

I would like to thank Dr. Amtoft Torben for being a member of my  M.S. committee, and for his excellent courses on algorithms and software specifications, which I enjoyed greatly and which were very beneficial to me. 

I would like to thank Dr. Mitchell L. Neilsen, for being a member of my M.S. committee, and for providing great advice and support throughout his classes, and for creating a very welcoming learning environment. 

I am also grateful to my family, friends and colleagues for their support and encouragement.



% +--------------------------------------------------------------------+
% | Dedication Page
% |
% | If you choose not to have a Dedication page, comment out
% | or delete the following 3 lines.
% +--------------------------------------------------------------------+

\phantomsection
\addcontentsline{toc}{chapter}{Dedication}
% +--------------------------------------------------------------------+
% | Dedication Page (Optional)
% +--------------------------------------------------------------------+

\newpage
\vspace*{0.9cm}
\begin{center}
{\bf \Huge Dedication}
\end{center}

\setlength{\baselineskip}{0.8cm}

%\pdfbookmark[0]{Dedication}{PDF_Dedication}

% +--------------------------------------------------------------------+
% | Enter the text for your dedication in the space below this box.    
% +--------------------------------------------------------------------+

I dedicate this work to my family.


% +--------------------------------------------------------------------+
% | This is where the chapter content of your ETDR begins.
% +--------------------------------------------------------------------+

%\phantomsection
\newpage
\pagenumbering{arabic}
\setcounter{page}{1}

% +--------------------------------------------------------------------+
% | Individual chapters of your ETDR are added using the \input
% | command.
% +--------------------------------------------------------------------+

\cleardoublepage

%---------------------------------------------------------------------------------------------------------------------
\chapter{Introduction}
\label{introduction}

In this chapter, we first introduce the basic terminology in the field of Machine Learning that we use thoughout this document in Section \ref{basicterminology}, and then provide background on disaster management in Section \ref{backgrounddisaster}.

We state the main problem addressed in Section \ref{problemdefinition}, where we also give a high-level overview of the approaches used in this work. 

%---------------------------------------------------------------------------------------------------------------------
\section{Basic Terminology}
\label{basicterminology}

An agent is learning if it improves its performance on future tasks after making observations
about the world, as it is defined in ~\citep{rn}. The examples of tasks may include: 
\begin{itemize}
  \item identify a given email as spam or non-spam
  \item predict housing prices for a given location
  \item determine if there is a specific object in a given image
  \item categorize news articles into topics such as politics, sports, entertainment, etc
\end{itemize}

Machine Learning algorithms use feature based representations for instances, where each instance is represented using a collection of features $f_1,f_2,\cdots,f_n$ ~\citep{tom}. An instance is a single object of the world from which a model will be learned, or on which a model will be used (e.g., for prediction). In most machine learning work, instances are described by feature vectors; some work uses more complex representations (e.g., containing relations between instances or between parts of instances) ~\citep{terms}. For example, an instance of the task "identify a given email as spam or non-spam" may be a text of an email represented as bag-of-words ~\citep{tom}. In this work, we use the words "instance" and "example" interchangeably. 

Two major types of learning are distinguished: supervised and unsupervised learning.

In supervised learning the agent is given a training set of N examples, which could be seen as input-–output pairs $(x_1, y_1), (x_2, y_2), \cdots(x_N, y_N)$, where each $y_j$ was generated by an unknown function $y = f(x)$. The task is to should discover a function $h$ that approximates the true function $f$ ~\citep{rn}. Identifying a given email as spam or non-spam is an example of a supervised learning task since training a model requires labeled instances, i.e. emails marked as spam and non-spam. 

In unsupervised learning the agent learns patterns in the input even though no explicit feedback is supplied ~\citep{rn}; essentially, it means that only $(x_1), (x_2), \cdots(x_N)$ are provided. Categorizing news articles into topics such as politics, sports, entertainment, etc is an example of an unsupervised learning task since it requires finding similarity between different news articles and clustering them together, with no prior labels provided. 

A classifier, or a classification model, is defined as a mapping from unlabeled instances to (discrete) classes. Classifiers have a form (e.g., decision tree) plus an interpretation procedure (including how to handle unknowns, etc.). Some classifiers also provide probability estimates (scores), which can be thresholded to yield a discrete class decision thereby taking into account a utility function ~\citep{terms}. 

In this work, we primarily focus on the first type of learning in the attempt to take advantage of labeled data. We also explore unsupervised methods to make use of unlabeled data. 


%---------------------------------------------------------------------------------------------------------------------
\section{Background on Disaster Management}
\label{backgrounddisaster}

Social media have become an integral part of disaster response. Twitter is one of the social media networks that can fill the void in areas where cell phone service might be lost, and where people look to resources to keep informed, locate loved ones, notify authorities and express support ~\citep{scientificamer}.

For instance, the Federal Emergency Management Agency (FEMA) wrote in its 2013 National Preparedness report that during and immediately following Hurricane Sandy, “users sent more than 20 million Sandy-related Twitter posts, or “tweets,” despite the loss of cell phone service during the peak of the storm.” New Jersey’s largest utility company, PSE\&G, said at the subcommittee hearing that during Sandy they staffed up their Twitter feeds and used them to send word about the daily locations of their giant tents and generators ~\citep{scientificamer}.

Following the Boston Marathon bombings, one quarter of Americans reportedly looked to Facebook, Twitter and other social networking sites for information, according to The Pew Research Center. The sites also formed a key part of the information cycle: when the Boston Police Department posted its final "CAPTURED!!!" tweet of the manhunt, more than 140,000 people retweeted it ~\citep{scientificamer}.

Furthermore, the National Disaster Management Authority (NDMA) spearheads an integrated approach to disaster management for the Government of India. They use Twitter to receive reports of damage from the field, and crowdsource the data to get sophisticated insights into what is happening in a region. Often the scale of events is so big, plotting data from Twitter helps to prioritize areas most affected. Twitter also helps the government communicate to the people affected what relief is available to them and where they can go to receive it ~\citep{twittercrisisblog}.

%---------------------------------------------------------------------------------------------------------------------
\section{Problem Definition}
\label{problemdefinition}

According to the members of the National Disaster Management Authority (NDMA), some of the challenges of using social media during disasters include tweaking the strategy in real time as ‘the disaster you planned for is not the disaster that happens’, using the right hashtags as part of a ‘good Twitter strategy', and devising ways to deal with misinformation and rumours ~\citep{twittercrisisblog}.

Machine learning methods may serve as an efficient solution to identify relevant tweets about disasters ~\citep{tweedr}. In fact, supervised machine learning methods have been used extensively because of the availability of labeled training data --- tweets about previous disasters --- that have been labeled by crowdsourcing, and thus, can be used to train a classification model ~\citep{starbird}. Furthermore, since tweets are primarily texts, natural language processing (NLP) methods for disaster management have been also researched by ~\citep{sakaki} and ~\citep{terpstra}. 

However, there are still many challenges in using relevant data from Twitter to help disaster response teams save people’s lives and property ~\citep{mendoza}. One of the major challenges is that for a current on-going disaster, no labeled data is available. Obviously, labeling data is an expensive, time-consuming and error-prone process. Thus, using supervised learning methods to aid disaster response may not be time-efficient. Yet, labeled data for a previous disaster may be available. We call a previous disaster data \textit{source} and current on-going disaster data \textit{target}. In addition, even though labeled target data is not available, unlabeled target data may still be available, and we explore its usage throughout this work.

We define the task as follows: given labeled source tweets, train a model to classify target tweets as \textit{relevant/not relevant} i.e. \textit{about disaster/not about disaster}. Yet, given that the distributions of source and target data are generally different, our model may not perform well. 

One of the key challenges is adapting the classifier to perform well on a target unlabeled domain. The process of adapting a classifier to make predictions on an unseen domain where labeled data is unavailable is called Domain Adaptation. In supervised machine learning, the general assumption is that both training and testing data come from the same distribution. This holds true for data coming from the saame domain. However, in real-world classification problems training and testing data may come from different domains. As a result, the performance of a classifier may drop significantly. Thus, it becomes essential to adapt the classifier trained on one domain to give accurate prediction on another domain. 

We raise the following questions, which we attempt to answer throughout this work:

\begin{itemize}
  \item is labeled source data sufficient to train a supervised learning model to make accurate predictions on target data?
  \item does single-source domain adaptation result in the higher accuracy?
  \item does multi-source domain adaptation result in the higher accuracy as compared to single-source domain adaptation?
\end{itemize}

In our experiments, we repeatedly take 2012 Sandy Hurricane, 2013 Queensland Floods, 2013 Boston Bombings, 2013 West Texas Explosion, 2013 Oklahoma Tornado and 2013 Alberta Floods tweets and combine them in source-target pairs based on the chronological order of the actual events. Our motivation is as follows: a \textit{source} disaster happens earlier than a \textit{target} disaster and thus, training a classifier in such a way complies more with future real-life applications.
\cleardoublepage

\chapter{Related work}
\label{relatedworkchapter}

In this chapter we discuss some of the previous work that has been done in the field of domain adaptation and review some of the relevant research papers.

Domain Adaptation has been researched in various machine learning applications, for instance, text classification ~\citep{dai}, bioinformatics ~\citep{eval}, cross-domain image retrieval ~\citep{crossdomimage}, multi-task learning ~\citep{multitaskdeep}, sentiment classification ~\citep{sentmulti}, among others. 

~\citet{dai} propose a novel transfer-learning algorithm for text classification based on an EM-based Naive Bayes classifiers. Their solution is to first estimate the initial probabilities under a distribution $D_l$ of one labeled data set, and then use an EM algorithm to revise the model for a different distribution $D_u$ of the test data which are unlabeled. According to ~\citep{dai}, the algorithm is very effective in several different pairs of domains, where the distances between the different distributions are measured using the Kullback--Leibler(KL) divergence. In their experiments, they show that the algorithm outperforms the traditional supervised and semi-supervised learning algorithms when the distributions of the training and test sets are increasingly different.

~\citet{eval} propose an approach for the task of splice site prediction. They use a weighted Naïve Bayes classifier, and analyze three methods for incorporating the target unlabeled data: EM with soft-labels, ST with hard-labels, and also a combination of EM/ST (with hard-labels for the most confidently labeled instances in the current target unlabeled data, and soft-labels for the other instances). They provide empirical results on splice site prediction indicating that using soft labels only can lead to better classifier compared to the other two ways.

~\citet{crossdomimage} propose a Dual Attribute-aware Ranking Network (DARN) for retrieval feature learning. DARN consists of two sub-networks, one for each domain with similar structure. Each of the two domain images are fed into each of the two sub-networks. As ~\citet{crossdomimage} states, the proposed method is different from previous approaches in that it simultaneously embeds semantic attribute information and visual similarity constraints into the feature learning stage, while modeling the discrepancy of the two domains. 

Similarly, ~\citet{multitaskdeep} adopts the deep learning approach, and developes a multi-task DNN for learning representations across multiple tasks. According to the authors, their multi-task DNN approach combines tasks of multiple-domain classification (for query classification) and information retrieval (ranking for web search), and 
shows better results over strong baselines in a comprehensive set of domain adaptation.

Also, the idea of learning from multiple sources is researched by ~\citet{sentmulti} in the area of sentiment classification. ~\citet{sentmulti} propose a new domain adaptation approach which can exploit sentiment knowledge from multiple source domains. They first extract both global and domain-specific sentiment knowledge from the data of multiple source domains using multi-task learning. Then, they transfer the knowledge from source domains to target domain with the help of words’ sentiment polarity relations extracted from the unlabeled target domain data. The authors state that experimental results show the effectiveness of the approach in improving cross-domain sentiment classification performance. However, their approach is not quite transferable to other problems. The reason is that it might be difficult to apply their method to other datasets because we would first need to build a sentiment word graph, on which the method heavily relies, and this is not scalable and not trivial. 

There has been some research done in the area of disaster management using tweets, by ~\citet{twitterda} and by ~\citet{imran16}, among others. 

~\citet{twitterda} study the usefulness of labeled data from a prior source disaster, together with unlabeled data from the current target disaster to learn domain adaptation classifiers for the target. Experimental results suggest that, for some tasks, source data itself can be useful for classifying target data. However, for tasks specific to a particular disaster, domain adaptation approaches that use target unlabeled data in addition to source labeled data are superior.

~\citet{imran16} research the performance of the classifiers trained using different combinations of training sets obtained from past disasters. They perform extensive experimentation on real crisis datasets and show that the past 
labels are useful when both source and target events are of the same type (e.g. both earthquakes). For similar languages, cross-language domain adaptation is useful, however, for different languages the performance decreases.

In this work, we first look closely at the approach described in ~\citep{coral} where they propose a simple, effective, and efficient method for unsupervised domain adaptation called CORrelation ALignment (CORAL), and use it in our single-source domain adaptation setting. CORAL aligns the input feature distributions of the source and target domains by exploring their second-order statistics. The method is "frustratingly easy" to implement: the only computation involved is recoloring the whitened source features with the covariance of the target domain. Extensive experiments on standard benchmarks demonstrate the superiority of their method over many existing state-of-the-art methods. These results confirm that CORAL is applicable to multiple features types, including highly performing deep features, and to different tasks, including computer vision and natural language processing.

Furthermore, we adopt the idea proposed in ~\citep{mda} and use it in our multi-source domain adaptation setting. ~\citet{mda} use causal models to represent the relationship between the features $X$ and class label $Y$, and consider possible situations where different modules of the causal model change with the domain. In each situation, they investigate what knowledge is appropriate to transfer and find the optimal target-domain hypothesis. They finally focus on the case where $Y$ is the cause for $X$ with changing $P_Y$ and $P_{X|Y}$, that is, $P_Y$ and $P_{X|Y}$ change independently across domains. Precisely, under appropriate assumptions, the availability of multiple source domains allows a natural way to reconstruct the conditional distribution on the target domain. They propose to model $P_{X|Y}$ (the process to generate effect X from cause Y) on the target domain as a linear mixture of those on source domains, and estimate all involved parameters by matching the target-domain feature distribution. According to ~\citep{mda}, experimental results on both synthetic and real-world data verify their theoretical results.












\cleardoublepage

%---------------------------------------------------------------------------------------------------------------------
\chapter{Data Description}
\label{datadescription}

In this chapter, we first present the original dataset in Section \ref{originaldataset}, and then describe the operations performed on the data (i.e., preprocessing) in Section \ref{datapreprocessing}. 

%---------------------------------------------------------------------------------------------------------------------
\section{Original Dataset}
\label{originaldataset}

In this work, we use the dataset CrisisLexT6 ~\citep{data}. The dataset consists of 60,000 tweets\footnote{A tweet is a short text (up to 140 characters) that users of Twitter can post on Twitter.com} posted during 6 crisis events in 2012 and 2013. The 60,000 tweets (10,000 in each disaster) have been labeled by crowdsourcing workers according to relatedness (as \textit{on-topic} or \textit{off-topic}). The \textit{on-topic} tweets are labeled as \textit{1}, and the \textit{off-topic} tweets are labeled as \textit{0}. The amount of tweets per class for each disaster is presented in Table \ref{originaldatasettable}.

%==================Table 1 begin
\begin{table}[ht]
    \begin{center}
    \caption{Original CrisisLeXT6 Dataset}
    \begin{tabular}[c]{|c|c|c|c|}
        \hline
        Crisis & On-topic & Off-topic & Total \\
        \hline
        2012 Sandy Hurricane & 6138 & 3870 & 10008 \\
        2013 Queensland Floods & 5414 & 4619 & 10033 \\ 
        2013 Boston Bombings & 5648 & 4364 & 10012 \\ 
        2013 West Texas Explosion & 5246 & 4760 & 10006 \\
        2013 Oklahoma Tornado & 4827 & 5165 & 9992 \\
        2013 Alberta Floods & 5189 & 4842 & 10031 \\
        \hline
    \end{tabular}
    \label{originaldatasettable}
   \end{center}
\end{table}
%==================Table 1 end


%---------------------------------------------------------------------------------------------------------------------
\section{Data Preprocessing}
\label{datapreprocessing}

The tweets are preprocessed before they are used in training, domain adaptation and testing stages.
The following cleaning steps have been taken ~\citep{twitterda}:

\begin{itemize}
  \item non-printable, ASCII characters are removed, as they are generally regarded as noise rather than useful information.
  \item printable HTML entities are converted into their corresponding ASCII equivalents
  \item URLs, email addresses, and usernames are replaced with a URL/email/username placeholder for each type of entity, respectively, under the assumption that those features could be predictive 
  \item numbers, punctuation signs and hashtags are kept under the assumption that numbers could be indicative of an address, while punctuation/emoticons and hashtags could be indicative of emotions 
  \item RT (i.e., retweet) are removed under the assumptions that such features are not informative for our classification tasks
  \item duplicate tweets and empty tweets (that have no characters left after the cleaning) are removed from the data sets
\end{itemize}

The number of tweets per class for each disaster was reduced by the preprocessing and is presented in Table \ref{aftercleaningdatasettable}.

%==================Table 1 begin
\begin{table}[ht]
    \begin{center}
    \caption{Dataset CrisisLeXT6 after preprocessing}
    \begin{tabular}[c]{|c|c|c|c|}
        \hline
        Crisis & On-topic & Off-topic & Total \\
        \hline
        2012 Sandy Hurricane & 5261 & 3752 & 9013 \\
        2013 Queensland Floods & 3236 & 4550 & 7786 \\ 
        2013 Boston Bombings & 4441 & 4309 & 8750 \\ 
        2013 West Texas Explosion & 4123 & 4733 & 8856 \\
        2013 Oklahoma Tornado & 3209 & 5049 & 8258 \\
        2013 Alberta Floods & 3497 & 4714 & 8211 \\
        \hline
    \end{tabular}
    \label{aftercleaningdatasettable}
   \end{center}
\end{table}
%==================Table 1 end

After preprocessing, the source tweets are expressed via target features i.e. via words that occur in the target tweets. The bag-of-words ~\citep{tom} representation is used to represent tweets as vectors of features. A sample bag-of-words tweet represenatation is presented in Table \ref{sampletweets}.


%==================Table 1 begin
\begin{table}[ht]
    \begin{center}
    \caption{Example of bag-of-words representation}
    \begin{tabular}[l]{|c|c|c|c|c|c|c|c|c|c|c|c|}
        \hline
        Sample Tweet & life & is & real & now & New & York & has & been & real & travel & to \\ 
        \hline
        Life is real now & 1 & 1 & 1 & 1 & 0 & 0 & 0 & 0 & 0 & 0 & 0 \\
        Travel to New York now & 0 & 0 & 0 & 1 & 1 & 1 & 0 & 0 & 0 & 1 & 1 \\
        New York has been real & 0 & 0 & 1 & 0 & 1 & 1 & 1 & 1 & 1 & 0 & 0 \\
        \hline
    \end{tabular}
    \label{sampletweets}
   \end{center}
\end{table}
%==================Table 1 end




\cleardoublepage

\chapter{Single-source Domain Adaptation Approach}
\label{coralchapter}

In this chapter, we define the problem of learning from a single source in Section ~\ref{coralproblemdefinitions}, and describe the correlation alignment algorithm ("CORAL") proposed in the paper ~\citep{coral} in Section ~\ref{coralalg}. Then we discuss the results obtained after applying CORAL in Section ~\ref{coralexperiments}.

%---------------------------------------------------------------------------------------------------------------------
\section{Problem Definition}
\label{coralproblemdefinitions}

We define our goal as follows: given tweets from a single source domain, train a model to classify tweets from a target domain. However, direct usage of source data may not give good performance, even if it is expressed via target features. So, we attempt to perform some transformations on source data to align its distribution with the target data distribution, assuming that labels for the target data are not available. As a possible solution, we adopt the method described in ~\citep{coral}, which minimizes the domain shift by aligning the second-order statistics of source and target distributions, without requiring any target labels. 



%---------------------------------------------------------------------------------------------------------------------
\section{Correlation Alignment Algorithm}
\label{coralalg}

~\citep{coral} present an extremely simple domain adaptation method --- CORrelation ALignment (CORAL) --- which works by aligning the distributions of the source and target features in an unsupervised manner. They propose to match the distributions by aligning the second-order statistics, namely, the covariance. More concretely, as ~\citep{coral} states, CORAL aligns the distributions by re-coloring whitened source features with the covariance of the target distribution. CORAL is simple and efficient, as the only computations it needs are:
\begin{itemize}
  \item computing covariance statistics in each domain
  \item applying the whitening and re-coloring linear transformation to the source features.
\end{itemize}

Then, supervised learning proceeds as usual -- training a classifier on the transformed source features.

They describe their method by taking a multi-class classification problem as the running example. Given source-domain training examples $D_S = \left \{ \overrightarrow{x_i} \right \}$, $\overrightarrow{x} \in \mathbb{R}^{D}$ with labels $L_S = \left \{ y_i \right \}$, $ y \in \left \{ {1, \cdots, L} \right \}$, and target data $D_T = \left \{ \overrightarrow{u_i} \right \}$, $\overrightarrow{u} \in \mathbb{R}^{D}$. Here both $\left \{ \overrightarrow{x} \right \}$ and $\left \{ \overrightarrow{u} \right \}$ are the $D$-dimensional feature representations $\varphi(I)$ of input $I$.

Suppose $\mu_s, \mu_t$ and $C_S, C_T$ are the feature vector means and covariance matrices. According to ~\citep{coral}, to minimize the distance between the second-order statistics (covariance) of the source and target features, they apply a linear transformation $A$ to the original source features and use the Frobenius norm as the matrix distance metric: \[ \min_{A} \left \| C_{\hat{S}} - C_T \right \| _{F}^{2} = \min_{A} \left \| A^{T}C_{S}A - C_T \right \| _{F}^{2} \] where $C_S$ is covariance of the transformed source features $D_{s}A$ and $\left \| \cdot   \right \| _{F}^{2}$ denotes the matrix Frobenius norm. Essentially, the solution lies in finding the matrix $A$.

After a series of calculations, which are presented in ~\citep{coral} in detail, the optimal solution can be found as: \[ A^{*} = U_{S}E \\ = (U_{S}\Sigma_{S}^{+\frac{1}{2}}U_{S}^\top)(U_{T[1:r]}\Sigma_{T[1:r]}^{+\frac{1}{2}}U_{T[1:r]}^\top) \] 
This can be interpreted as follows: the first part whitens the source data while the second part re-colors it with the target covariance. 

\textit{Whitening} refers to the process of first de-correlating the data $y$ -- its covariance, $\mathbf{E}(yy^\top)$ is now a diagonal matrix, $\Lambda$. The diagonal elements (eigenvalues) in $\Lambda$ may be the same or different. If we make them all the same, then this is called whitening the data ~\citep{rosalind}. 

\textit{Re-coloring} generally refers to the process of transforming a vector of white random variables into a random vector with a specified covariance matrix ~\citep{miliha}.

As ~\citep{coral} suggests, after CORAL transforms the source features to the target space, a classifier $f_{\overrightarrow{w}}$ parametrized by $\omega$ can be trained on the adjusted source features and directly applied to target features. 

Since correlation alignment changes the features only, it can be applied to any base classifier. In this work, we run experiments using Naive Bayes Classifier. 

The first supervised learning method we use is the multinomial Naive Bayes or multinomial NB model, a probabilistic learning method ~\citep{ir}. It is used for multinomially distributed data, for instance, in text classification where the data are typically represented as word vector counts. We use it with \textit{counts} representation for the data.

An alternative to the multinomial model is the multivariate Bernoulli model or Bernoulli model. It generates an indicator for each term of the vocabulary, either $1$ indicating presence of the term in the document or $0$ indicating absence ~\citep{ir}. We use it with \textit{0/1} representation for the data.

However, since CORAL changes the values of the data from \textit{0/1} to continuous, we need to use Gaussian Naive Bayes algorithm for classification of such data. The likelihood of the features is assumed to be Gaussian: \[ P(x_i|y) = \frac{1}{\sqrt{2\pi \sigma ^{2}_y}} exp\left ( - \frac{(x_i - \mu_y)^2}{2\sigma ^2_y} \right ) \] 

The results of the experiments are discussed in Section ~\ref{coralexperiments}.


%---------------------------------------------------------------------------------------------------------------------
\section{Experiments and Results}
\label{coralexperiments}

We setup the experiments as follows:

\begin{itemize}
  \item use only target features to represent sources
  \item perform 5-fold cross-validation over target and report the average accuracy over the 5 folds (each source is "aligned" with 3 target unlabeled folds using CORAL, 1 labeled target fold is used for testing, 1 labeled target fold is kept for possible use of labeled target data)
  \item vary the number of instances in the sources (i.e. 500 instances per class, 1000 instances per class, etc) - smaller datasets are subsets of th larger datasets
\end{itemize}


%---------------------------------------------------------------------------------------------------------------------
\subsection{Preliminary Results without Domain Adaptation}
\label{preresults}

We present the results obtained when no domain adaptation is performed. The source data is expressed via target features and is described in Table \ref{table1}. Then, two representations are tested: \textit{0/1} and \textit{counts} to determine which representation is more promising i.e. gives better results. Bernoulli Naive Bayes and Multinomial Naive Nayes classifiers are used for \textit{0/1} and \textit{counts} representations respectively. 

The target data is divided into five folds for cross-validation ~\citep{hastie}. Each fold in turn is used for testing, and the accuracy is recorded in each run. The average results are reported. The number of instances per class in the source data is varied: the results are recorded for source data having $500$ instances per class, $1000$ instances per class, and $2000$ instances per class. All source data is also used as training data resulting in slightly better accuracy for $Source 2$ and $Source  3$ ($0.7888$ and $0.7421$, respectively, as compared to $0.7810$ and $0.7346$, respectively, for the number of instances equal to $2000$) in Table \ref{table3}. Generally, balanced training data (i.e. data that has equal number of instances per class) gives better performance.

The results in Table \ref{table3} and Table \ref{table4} indicate that \textit{0/1} representation is more efficient. One of the possible explanations might be that it is unlikely that meaningful words are repeatedly used in a single tweet since tweets are relatively short. Thus, keeping the actual counts does not provide additional relevant information.


%==================Table 1 begin
\begin{table}[ht]
    \begin{center}
    \caption{Target and Source datasets}
    \begin{tabular}[c]{|c|c|c|c|}
        \hline
        Target & Source 1 & Source 2 & Source 3 \\
        \hline
        2013 Alberta Floods & 2012 Sandy Hurricane & 2013 Queensland Floods & 2013 Boston Bombings \\
        \hline
    \end{tabular}
    \label{table1}
   \end{center}
\end{table}
%==================Table 1 end


%==================Table 3
\begin{table}[ht]
    \begin{center}

% +--------------------------------------------------------------------+
% | The table is created with this command
% |
% | \begin{tabular}[pos]{table spec}
% |
% | The "pos" argument specifies the vertical position of the table
% | relative to the baseline of the surrounding text.  Use t, b, or c
% | to specify alignment at the top, bottom, or center.
% |
% | The "table spec" command defines the format of the table
% |   l for a column of left-aligned text
% |   r for a column of right-aligned text
% |   c for centered text
% |   p{width} for a column containing justified text with line breaks
% |   | for a vertical line
% |
% |  In this example, the caption is made to appear above the table
% |  by positioning the \caption command before the \begin{tabular
% |  command. To position the caption below the table, insert the
% |  \caption command after the \end{tabular} command.
% +--------------------------------------------------------------------+

    \caption{Accuracy after running Bernoulli Naive Bayes with 0/1 representation for instances}
    \begin{tabular}[c]{|c|c|c|c|c|}
        \hline
        Sources & 500 & 1000 & 2000 & All \\
        \hline
        Source 1 & 0.699183336 & 0.737669795 & 0.748874085 & 0.714163926 \\
        Source 2 & 0.759834769 & 0.764704356 & 0.781025693 & 0.788819876 \\
        Source 3 & 0.710511727 & 0.716357069 & 0.734625396 & 0.742175131 \\
        \hline
    \end{tabular}

    \label{table3}
   \end{center}
\end{table}
%==================Table 3 end


%==================Table 4 end
\begin{table}[ht]
    \begin{center}
    \caption{Accuracy after running Multinomial Naive Bayes with counts representation for instances}
    \begin{tabular}[c]{|c|c|c|c|c|}
        \hline
        Sources & 500 & 1000 & 2000 & All \\
        \hline
        Source 1 & 0.667883087 & 0.708560882 & 0.727195729 & 0.694068932 \\
        Source 2 & 0.749968752 & 0.758614519 & 0.772012294 & 0.786384119 \\
        Source 3 & 0.676287695 & 0.697966125 & 0.715016795 & 0.720984046 \\
        \hline
    \end{tabular}

    \label{table4}
   \end{center}
\end{table}
%==================Table 4 end

After running the preliminary experiments both with \textit{0/1} representation and \textit{counts} representation using Naive Bayes Classifier ~\citep{tom}, the results obtained in Table \ref{table3} have shown that \textit{0/1} representation gives better performance as compared to the \textit{counts} representation results presented in Table \ref{table4}. Consequently, in the further experiments the \textit{0/1} representation is used.


%==================Table begin
\begin{table}[ht]
    \begin{center}
    \caption{Accuracy after applying CORAL with Gaussian Naive Bayes}
    \begin{tabular}[c]{|c|c|c|c|c|}
        \hline
        Source Disaster Event & 500 & 1000 & 2000 & Total \\
        \hline
        2012 Sandy Hurricane & 0.670077908 & 0.733406627 & 0.77286521 & 0.797223237 \\
        2013 Queensland Floods & 0.666543406 & 0.645719596 & 0.684447955 & 0.657191572\\
        2013 Boston Bombings & 0.643529372 & 0.573255749 & 0.599684114 & 0.649713799 \\
        \hline
    \end{tabular}
    \label{tablecoral}
   \end{center}
\end{table}
%==================Table end

We can see improvement for one pair of source, precisely, 2012 Sandy Hurricane. After applying CORAL, the accuracy increases from $0.74$ to $0.77$ for a subset of $2000$ instances per class and from $0.71$ to $0.79$ for a set that uses all instances. In other cases, the accuracy decreases. One of the possible reasons might be that the initial matrix is very sparse and has a large number of features (precisely, $1334$) as compared to the number of instances ($1000, 2000$ etc). In the original paper ~\citep{coral}, they reduce the original dimentionaly of the dataset to keep top $400$ features based on mutual information ~\citep{hastie}. 

Similarly, we attempt to reduce the dimentionaly of our dataset. We experiment with several dimentionaly reduction techniques. Since we assume that target labeled data is not available, using mutual information for feature selection is not quite applicable.

\subsection{Feature Selection}
\label{varthressub}

%-------------------------------------------------------------------------------------
\subsubsection{Principle Component Analysis}
One of the dimentionality reduction methods that does not require labeled data is Principle Component Analysis (PCA) -- standard linear principal components are obtained from the eigenvectors of the covariance matrix, and give directions in which the data have maximal variance ~\citep{hastie}. Basically, we select $k$ principle components that should describe our data well. One of the recommended methods to choose $k$ is to choose the smallest $k$ for which $99\%$ of variance retained. However, when we choose $k$ in this way, the number of components (i.e. features) retained becomes $1332$, which is only $2$ features less than the original dimentionality of $1334$. Therefore, we decide to not proceed with using PCA on our data.

%-------------------------------------------------------------------------------------
\subsubsection{Variance Threshold}
\label{varthressub}

Feature selector that removes all low-variance features. 

This feature selection algorithm looks only at the features $X$, not the desired outputs $y$, and can thus be used for unsupervised learning ~\citep{varthres}. Specifically, we have a dataset with boolean features, and we want to remove all features that are either one or zero (on or off), for instance, in more than $99\%$ of the samples. Boolean features are Bernoulli random variables, and the variance of such variables is given by: \[ Var[X] = p(1-p)\] so we can select features using the threshold equal to $.99 * (1 - .99)$.

In order to select features, we first concatenate source data and target data, which we call \textit{training target unlabeled -- tTU}. Second, we transform source data and tTU using the extracted features. Then, we run CORAL on source data and tTU. We fit the classifier with transformed source data after the CORAL stage, and we test on \textit{testing target data}.

Features with a training-set variance lower than the specified threshold are removed. We apply feature selection for every combination of source-target fold, so the number of features varies from $160$ to $176$. That is a significant reduction (by $88\%$) compared to the original dimentionaly of the data -- $1334$. 

The value of threshold is varied across the experiments. Precisely, we experiment with $k=0.95, k=0.90, k=0.80$, and we present results in Table \ref{tablevar95}, in Table \ref{tablevar9}, in Table \ref{tablevar8} respectively. The highest accuracy is obtained when the threshold is equal to $0.99$ as described in Table \ref{tablevar99}. 

We decide to further check whether feature selection by itself improves performance, without applying CORAL. We run select features based on Varience Threshold equal to $0.99$ and then run the Bernoulli Naive Bayes classifier. The results presented in Table \ref{tablevar99nocoral} show that applying CORAL does improve performance. Thus, eliminating features with variance lower than $0.99$ and then applying CORAL is more efficient than when CORAL is not applied. 

%==================Table begin
\begin{table}[ht]
    \begin{center}
    \caption{Accuracy after selecting features via Variance Threshold (0.99) without applying CORAL with Bernoulli Naive Bayes}
    \begin{tabular}[c]{|c|c|c|c|c|}
        \hline
        Source Disaster Event & 500 & 1000 & 2000 & Total \\
        \hline
        2012 Sandy Hurricane & 0.669101929 & 0.701010747 & 0.715869636 & 0.651807358 \\
        2013 Queensland Floods & 0.740105627 & 0.740835257 & 0.742905828 & 0.74838643 \\
        2013 Boston Bombings & 0.685786895 & 0.694311377 & 0.69711321 & 0.695408269 \\
        \hline
    \end{tabular}
    \label{tablevar99nocoral}
   \end{center}
\end{table}
%==================Table end


%==================Table begin
\begin{table}[ht]
    \begin{center}
    \caption{Accuracy after selecting features via Variance Threshold (0.99) and applying CORAL with Gaussian Naive Bayes}
    \begin{tabular}[c]{|c|c|c|c|c|}
        \hline
        Source Disaster Event & 500 & 1000 & 2000 & Total \\
        \hline
        2012 Sandy Hurricane & 0.770675727 & 0.843014879 & 0.857263347 & 0.846181082 \\
        2013 Queensland Floods & 0.732553416 & 0.807941416 & 0.813179302 & 0.802460296 \\
        2013 Boston Bombings & 0.712096496 & 0.712092048 & 0.744128673 & 0.791619412 \\
        \hline
    \end{tabular}
    \label{tablevar99}
   \end{center}
\end{table}
%==================Table end



%==================Table begin
\begin{table}[ht]
    \begin{center}
    \caption{Accuracy after selecting features via Variance Threshold (0.95) and applying CORAL with Gaussian Naive Bayes}
    \begin{tabular}[c]{|c|c|c|c|c|}
        \hline
        Source Disaster Event & 500 & 1000 & 2000 & Total \\
        \hline
        2012 Sandy Hurricane & 0.589452614 & 0.775300967 & 0.735233297 & 0.652419188 \\
        2013 Queensland Floods & 0.837409658 & 0.81122223 & 0.837044324 & 0.698949665 \\
        2013 Boston Bombings & 0.639519076 & 0.706117934 & 0.781025248 & 0.734260358 \\
        \hline
    \end{tabular}
    \label{tablevar95}
   \end{center}
\end{table}
%==================Table end

%==================Table begin
\begin{table}[ht]
    \begin{center}
    \caption{Accuracy after selecting features via Variance Threshold (0.90) and applying CORAL with Gaussian Naive Bayes}
    \begin{tabular}[c]{|c|c|c|c|c|}
        \hline
        Source Disaster Event & 500 & 1000 & 2000 & Total \\
        \hline
        2012 Sandy Hurricane & 0.703077612 & 0.681526025 & 0.639758233 & 0.592012176 \\
        2013 Queensland Floods & 0.815486955 & 0.754715498 & 0.734633995 & 0.657777097 \\
        2013 Boston Bombings & 0.743512617 & 0.700768106 & 0.679461385 & 0.626111366 \\
        \hline
    \end{tabular}
    \label{tablevar9}
   \end{center}
\end{table}
%==================Table end

%==================Table begin
\begin{table}[ht]
    \begin{center}
    \caption{Accuracy after selecting features via Variance Threshold (0.80) and applying CORAL with Gaussian Naive Bayes}
    \begin{tabular}[c]{|c|c|c|c|c|}
        \hline
        Source Disaster Event & 500 & 1000 & 2000 & Total \\
        \hline
        2012 Sandy Hurricane & 0.561565954 & 0.590795039 & 0.585071351 & 0.517477091 \\
        2013 Queensland Floods & 0.62684663 & 0.616980465 & 0.636341457 & 0.644623001 \\
        2013 Boston Bombings & 0.743512617 & 0.700768106 & 0.679461385 & 0.626111366 \\
        \hline
    \end{tabular}
    \label{tablevar8}
   \end{center}
\end{table}
%==================Table end

We decide to test the method further on all pairs of source and target disasters described in Table ~\ref{tableadddata}. The results of applying feature selection followed by CORAL are shown in Table ~\ref{tablevar99adddata}, and the results of applying feature selection without CORAL are presented in Table ~\ref{tablevar99adddatanocoral}. 

In addition, we present the number of features retained after applying Variance Threshold to the pair 2013 Boston Bombings -- 2013 Alberta Floods when the different amount of the source instances is used in Figure \ref{featuresretained99figure}. Precisely, $1000$ number of the source instances means that there are $500$ instances of the class \textit{1}, and $500$ instances of the class \textit{0}, the same logic applies to $2000$ and $4000$. $8750$ instances means that we use all the instances in the source data, and it becomes imbalanced. The plot gives a better understanding of the effect of Variance Threshold on reducing the feature space. The number of features for other pairs is in a similar range of $159-187$, which varies for different pairs and their folds.

Interestingly, when the threshold is set to $0.95$, the number of features retained reduces even further, and is presented in Figure \ref{featuresretained95figure}. 


%==================Table begin
\begin{table}[ht]
    \begin{center}
    \caption{All Pairs of Source-Target Disasters}
    \begin{tabular}[c]{|c|c|c|}
        \hline
        Pair & Source Disaster Event & Target Disaster Event  \\
        \hline
        $SH \rightarrow QF$ & 2012 Sandy Hurricane & 2013 Queensland Floods \\

        $SH \rightarrow BB$ & 2012 Sandy Hurricane & 2013 Boston Bombings \\
        $QF \rightarrow BB$ & 2013 Queensland Floods & 2013 Boston Bombings \\

        $SH \rightarrow WT$ & 2012 Sandy Hurricane & 2013 West Texas Explosion \\
        $BB \rightarrow WT$ & 2013 Boston Bombings & 2013 West Texas Explosion \\

        $SH \rightarrow OT$ & 2012 Sandy Hurricane & 2013 Oklahoma Tornado \\
        $QF \rightarrow OT$ & 2013 Queensland Floods & 2013 Oklahoma Tornado \\
        $BB \rightarrow OT$ & 2013 Boston Bombings & 2013 Oklahoma Tornado  \\

        $SH \rightarrow AF$ & 2012 Sandy Hurricane & 2013 Alberta Floods \\
        $QF \rightarrow AF$ & 2013 Queensland Floods & 2013 Alberta Floods \\
        $BB \rightarrow AF$ & 2013 Boston Bombings & 2013 Alberta Floods \\
        \hline
    \end{tabular}
    \label{tableadddata}
   \end{center}
\end{table}
%==================Table end

%==================Table begin
\begin{table}[ht]
    \begin{center}
    \caption{Accuracy after selecting features via Variance Threshold (0.99) without applying CORAL with Bernoulli Naive Bayes. All pairs}
    \begin{tabular}[c]{|c|c|c|c|c|c|}
        \hline
        Pair & 500 & 1000 & 2000 & Total \\
        \hline
        $SH \rightarrow QF$ & 0.645647344  &    0.76355232  &  0.773312788  & 0.724891438 \\ %

        $SH \rightarrow BB$ & 0.694285714   &  0.7664  &  0.703314286  & 0.6864 \\ %
        $QF \rightarrow BB$ & 0.712685714   &  0.703314286  & 0.716342857  & 0.717485714 \\ %

        $SH \rightarrow WT$ & 0.682135879  &   0.735997122  & 0.714882615  & 0.738481467 \\ %
        $BB \rightarrow WT$ & 0.923893032   &  0.931683902  & 0.93157129   & 0.94241173 \\ %

        $SH \rightarrow OT$ & 0.763017791   &  0.768710137  & 0.795108695  & 0.762413197 \\ %
        $QF \rightarrow OT$ & 0.796802144   &  0.801283322  & 0.815331232  & 0.815209727 \\ %
        $BB \rightarrow OT$ & 0.796561919   &  0.791354935  & 0.806612765  & 0.808791942 \\ %

        $SH \rightarrow AF$ & 0.669101929 & 0.701010747 & 0.715869636 & 0.651807358 \\ %
        $QF \rightarrow AF$ & 0.740105627 & 0.740835257 & 0.742905828 & 0.74838643 \\  %
        $BB \rightarrow AF$ & 0.685786895 & 0.694311377 & 0.69711321 & 0.695408269 \\ %


        \hline
    \end{tabular}
    \label{tablevar99adddatanocoral}
   \end{center}
\end{table}
%==================Table end

%==================Table begin
\begin{table}[ht]
    \begin{center}
    \caption{Accuracy after selecting features via Variance Threshold (0.99) and applying CORAL with Gaussian Naive Bayes. All pairs}
    \begin{tabular}[c]{|c|c|c|c|c|c|}
        \hline
        Pair & 500 & 1000 & 2000 & Total \\
        \hline
        $SH \rightarrow QF$ & 0.751221161 & 0.851656398 & 0.852042661 & 0.838684627 \\

        $SH \rightarrow BB$ & 0.789714286 & 0.764114286 & 0.827314286 & 0.766971429 \\
        $QF \rightarrow BB$ & 0.834057143 & 0.819657143 & 0.804571429 & 0.68 \\

        $SH \rightarrow WT$ & 0.802055629 & 0.738821915 & 0.674232334 & 0.836269315 \\
        $BB \rightarrow WT$ & 0.888098892 & 0.945347159 & 0.94568627 & 0.949412532 \\

        $SH \rightarrow OT$ & 0.85359559 & 0.858683049 & 0.857835445 & 0.753334467\\
        $QF \rightarrow OT$ & 0.82743645 & 0.867645187 & 0.873215221 & 0.815452151 \\
        $BB \rightarrow OT$ & 0.792924312 & 0.853962013 & 0.827806759 & 0.823568371 \\

        $SH \rightarrow AF$ & 0.770675727 & 0.843014879 & 0.857263347 & 0.846181082 \\
        $QF \rightarrow AF$ & 0.732553416 & 0.807941416 & 0.813179302 & 0.802460296 \\
        $BB \rightarrow AF$ & 0.712096496 & 0.712092048 & 0.744128673 & 0.791619412 \\

        \hline
    \end{tabular}
    \label{tablevar99adddata}
   \end{center}
\end{table}
%==================Table end

%==================Figure begin
\begin{figure}[ht]
\centering
\caption{Number of features retained after applying Variance Threshold (0.99)}
\begin{tikzpicture}
\begin{axis}[
    title={Source -- 2013 Boston Bombings, Target -- 2013 Alberta Floods},
    xlabel={Number of source instances},
    ylabel={Number of features retained},
    width=0.8\textwidth,
    % xmin=900, xmax=10000,
    % ymin=150, ymax=190,
    xtick={1000, 2000, 4000, 8750},
    % ytick={155, 165, 175, 185},
    legend pos=north west,
    ymajorgrids=true,
    grid style=dashed,
]
 
\addplot[only marks,
    color=blue,
    mark = *
    %mark=square,
    ]
    table {vt_features_99.dat};
 
\end{axis}
\end{tikzpicture}
\label{featuresretained99figure}
\end{figure}
%==================Figure end


%==================Figure begin
\begin{figure}[ht]
\centering
\caption{Number of features retained after applying Variance Threshold (0.95)}
\begin{tikzpicture}
\begin{axis}[
    title={Source -- 2013 Boston Bombings, Target -- 2013 Alberta Floods},
    xlabel={Number of source instances},
    ylabel={Number of features retained},
    width=0.8\textwidth,
    % xmin=900, xmax=10000,
    % ymin=150, ymax=190,
    xtick={1000, 2000, 4000, 8750},
    % ytick={155, 165, 175, 185},
    legend pos=north west,
    ymajorgrids=true,
    grid style=dashed,
]
 
\addplot[only marks,
    color=blue,
    mark = *
    %mark=square,
    ]
    table {vt_features_95.dat};
 
\end{axis}
\end{tikzpicture}
\label{featuresretained95figure}
\end{figure}
%==================Figure end


%==================Figure begin
\begin{figure}[ht]
\centering
\caption{Number of features retained after applying Variance Threshold (0.90)}
\begin{tikzpicture}
\begin{axis}[
    title={Source -- 2013 Boston Bombings, Target -- 2013 Alberta Floods},
    xlabel={Number of source instances},
    ylabel={Number of features retained},
    width=0.8\textwidth,
    % xmin=900, xmax=10000,
    % ymin=150, ymax=190,
    xtick={1000, 2000, 4000, 8750},
    % ytick={155, 165, 175, 185},
    legend pos=north west,
    ymajorgrids=true,
    grid style=dashed,
]
 
\addplot[only marks,
    color=blue,
    mark = *
    %mark=square,
    ]
    table {vt_features_90.dat};
 
\end{axis}
\end{tikzpicture}
\label{featuresretained90figure}
\end{figure}
%==================Figure end


We decide to further experiment with this feature selection method, precisely, we now rank features by their variance i.e. calcucating $p(1-p)$ for each feature across all samples, in the descending order. We then take top $k$ features and run the experiments again. The results for $k = 190$ without applying CORAl and with applying CORAL are shown in Table \ref{top190vartablenocoral} and in Table \ref{top190vartablecoral}, respectively, and the results for $k = 170$ with applying CORAL are shown in Table \ref{top170vartablecoral}.

%==================Table begin
\begin{table}[ht]
    \begin{center}
    \caption{Accuracy after ranking features by variance in the descending order, and selecting top k features (k=190), without applying CORAL with Bernoulli Naive Bayes. All pairs}
    \begin{tabular}[c]{|c|c|c|c|c|c|}
        \hline
        Pair & 500 & 1000 & 2000 & Total \\
        \hline
        $SH \rightarrow QF$ &  0.650399414   & 0.77138584   & 0.78179063   & 0.730799495 \\

        $SH \rightarrow BB$ &  0.693485714   & 0.766171429  & 0.7048   & 0.692342857 \\
        $QF \rightarrow BB$ &  0.709714286   & 0.710057143  & 0.718628571  & 0.725714286 \\

        $SH \rightarrow WT$ &  0.681909635  & 0.735659031 & 0.7162381 & 0.73915905 \\
        $BB \rightarrow WT$ &  0.924231696   & 0.931345237  & 0.932022884  & 0.942976255 \\

        $SH \rightarrow OT$ &  0.766046699   & 0.775854974  & 0.804795391  & 0.778155509 \\
        $QF \rightarrow OT$ &  0.801041925   & 0.804068706  & 0.818116396  & 0.823080443 \\
        $BB \rightarrow OT$ &  0.798257348   & 0.790748508  & 0.807339011  & 0.81714736 \\

        $SH \rightarrow AF$ &  0.67555547    & 0.710753775  & 0.718305616  & 0.658506134 \\
        $QF \rightarrow AF$ &  0.744489856   & 0.742174641  & 0.746437142  & 0.752405103\\
        $BB \rightarrow AF$ &  0.694069181   & 0.70393327   & 0.702228626  & 0.693703105 \\

        \hline
    \end{tabular}
    \label{top190vartablenocoral}
   \end{center}
\end{table}
%==================Table end


%==================Table begin
\begin{table}[ht]
    \begin{center}
    \caption{Accuracy after ranking features by variance in the descending order, selecting top k features (k=190), and applying CORAL with Gaussian Naive Bayes. All pairs}
    \begin{tabular}[c]{|c|c|c|c|c|c|}
        \hline
        Pair & 500 & 1000 & 2000 & Total \\
        \hline
        $SH \rightarrow QF$ & 0.78114606  & 0.858591907 & 0.860391309 & 0.830336061    \\ %
                              
        $SH \rightarrow BB$ & 0.793714286 & 0.7704      & 0.818742857 & 0.766857143    \\ %
        $QF \rightarrow BB$ & 0.753828571 & 0.791085714 & 0.810285714 & 0.670628571    \\%
                               
        $SH \rightarrow WT$ & 0.791216081 & 0.741531547 & 0.686653802 & 0.832543181     \\%
        $BB \rightarrow WT$ & 0.883921163 & 0.945686079 & 0.948508832 & 0.95076769     \\%
                               
        $SH \rightarrow OT$ & 0.850326605 & 0.866917695 & 0.860378551 & 0.770045156    \\%
        $QF \rightarrow OT$ & 0.806971048 & 0.86643248  & 0.867402763 & 0.827076627    \\%
        $BB \rightarrow OT$ & 0.792201513 & 0.851418247 & 0.81460484  & 0.827200625     \\%
                               
        $SH \rightarrow AF$ & 0.771767948 & 0.834732001 & 0.863352517 & 0.853122574  \\ %
        $QF \rightarrow AF$ & 0.722325697 & 0.806237216 & 0.816832789 & 0.759592573 \\  %
        $BB \rightarrow AF$ & 0.652545809 & 0.705151075 & 0.70296389  & 0.771159157  \\%

        \hline
    \end{tabular}
    \label{top190vartablecoral}
   \end{center}
\end{table}
%==================Table end

%==================Table begin
\begin{table}[ht]
    \begin{center}
    \caption{Accuracy after ranking features by variance in the descending order, selecting top k features (k=170), and applying CORAL with Gaussian Naive Bayes. All pairs}
    \begin{tabular}[c]{|c|c|c|c|c|c|}
        \hline
        Pair & 500 & 1000 & 2000 & Total \\
        \hline
        $SH \rightarrow QF$ & 0.744414351 & 0.842666149 & 0.854740156   & 0.83958338   \\ %
                             
        $SH \rightarrow BB$ & 0.789714286 & 0.763085714 & 0.827428571   & 0.765942857   \\ %
        $QF \rightarrow BB$ & 0.812914286 & 0.8128  & 0.808457143       & 0.667657143   \\%
                             
        $SH \rightarrow WT$ & 0.80826528  & 0.739837334 & 0.676491072 & 0.834688224    \\%
        $BB \rightarrow WT$ & 0.886856528 & 0.944217854 & 0.945460409 & 0.949525462   \\%
                             
        $SH \rightarrow OT$ & 0.852627361 & 0.858683049 & 0.858319706 & 0.75309219   \\%
        $QF \rightarrow OT$ & 0.835914399 & 0.866071557 & 0.868976759 & 0.823080516   \\%
        $BB \rightarrow OT$ & 0.793046184 & 0.84390794  & 0.824658106 & 0.827927604    \\%
                             
        $SH \rightarrow AF$ & 0.76263675  & 0.8427712   & 0.858968584 & 0.845937625 \\ %
        $QF \rightarrow AF$ & 0.73230981  & 0.807819613 & 0.814884317 & 0.791254449 \\  %
        $BB \rightarrow AF$ & 0.718186778 & 0.711848591 & 0.738403725 & 0.788696593 \\%

        \hline
    \end{tabular}
    \label{top170vartablecoral}
   \end{center}
\end{table}
%==================Table end

%-------------------------------------------------------------------------------------
\subsubsection{Truncated SVD aka Latent Semantic Analysis}

This transformer performs linear dimensionality reduction by means of truncated singular value decomposition (SVD) ~\citep{ir}. Contrary to PCA, this estimator does not center the data before computing the singular value decomposition. This means it can work with scipy.sparse matrices efficiently. 

The results on our data show that Truncated SVD does not contribute to better accuracy. We manually choose the number of components $k$. Concretely, we experiment with $k=677$, choosing this values as $50\%$ of the original number of features ($1334$), with $k=400$ and with $k=170$, choosing this value as an average number of features obtained after Variance Threshold discussed in \ref{varthressub} that give the best performance so far.

%==================Table begin
\begin{table}[!h]%[ht]
    \begin{center}
    \caption{Accuracy after selecting features via Truncated SVD (k=677) and applying CORAL with Gaussian Naive Bayes}
    \begin{tabular}[c]{|c|c|c|c|c|}
        \hline
        Source Disaster Event & 500 & 1000 & 2000 & Total \\
        \hline
        2012 Sandy Hurricane & 0.610885735 & 0.61076534 & 0.631833052 & 0.614783494 \\
        2013 Queensland Floods & 0.644013172 & 0.648520761 & 0.651077431 & 0.646450264 \\
        2013 Boston Bombings & 0.635488616 & 0.58519041 & 0.606746297 & 0.609668153 \\
        \hline
    \end{tabular}
    \label{tabletrunsvd677}
   \end{center}
\end{table}
%==================Table end

%==================Table begin
\begin{table}[!h]%[ht]
    \begin{center}
    \caption{Accuracy after selecting features via Truncated SVD (k=400) and applying CORAL with Gaussian Naive Bayes}
    \begin{tabular}[c]{|c|c|c|c|c|}
        \hline
        Source Disaster Event & 500 & 1000 & 2000 & Total \\
        \hline
        2012 Sandy Hurricane & 0.629033148 & 0.630251842 & 0.64462226 & 0.634392095 \\
        2013 Queensland Floods & 0.656436971 & 0.617708835 & 0.626234503 & 0.606870175 \\
        2013 Boston Bombings & 0.646330092 & 0.56668048 & 0.578736203 & 0.582876604 \\
        \hline
    \end{tabular}
    \label{tabletrunsvd400}
   \end{center}
\end{table}
%==================Table end

%==================Table begin
\begin{table}[!h]%[ht]
    \begin{center}
    \caption{Accuracy after selecting features via Truncated SVD (k=170) and applying CORAL with Gaussian Naive Bayes}
    \begin{tabular}[c]{|c|c|c|c|c|}
        \hline
        Source Disaster Event & 500 & 1000 & 2000 & Total \\
        \hline
        2012 Sandy Hurricane & 0.610885735 & 0.61076534 & 0.631833052 & 0.614783494 \\
        2013 Queensland Floods & 0.644013172 & 0.648520761 & 0.651077431 & 0.646450264 \\
        2013 Boston Bombings & 0.635488616 & 0.58519041 & 0.606746297 & 0.609668153 \\
        \hline
    \end{tabular}
    \label{tabletrunsvd170}
   \end{center}
\end{table}
%==================Table end


%-------------------------------------------------------------------------------------
\subsubsection{Mutual Information}

The mutual information of two random variables is a natural measure of dependence between the two variables ~\citep{hastie}, which can be expressed as follows: \[I(x,y) = \sum_{x,y} P(x,y) \ln {{P(x,y)}\over{P(x) P(y)}} \]

Our first experiment in this subsection we setup as follows:
  \begin{itemize}
  \item express source via target features
  \item select top K features from source based on Mutual Information
  \item transform the source and target validation fold to be expressed via newly obtained top $K$ features 
  \item run CORAL on transformed source and transformed target obtained at the previous step
  \item fit the classifier with the transformed source obtained at the previous step
  \item transform the target test fold to be expressed via top $K$ features
  \item test the classifier on target test fold
  \end{itemize}

We select $k=300$ and present the results in Table \ref{tablemisource300}. 

%==================Table begin
\begin{table}[ht]
    \begin{center}
    \caption{Accuracy after selecting features based on Mutual Information in Source (k=300) and applying CORAL with Gaussian Naive Bayes}
    \begin{tabular}[c]{|c|c|c|c|c|}
        \hline
        Source Disaster Event & 500 & 1000 & 2000 & Total \\
        \hline
        2012 Sandy Hurricane & 0.668858769 & 0.710877135 & 0.648762884 & 0.698939879 \\
        2013 Queensland Floods & 0.704055592 & 0.742418098 & 0.708438635 & 0.63378575 \\
        2013 Boston Bombings & 0.46681585 & 0.6857886 & 0.500670619 & 0.690170383 \\
        \hline
    \end{tabular}
    \label{tablemisource300}
   \end{center}
\end{table}
%==================Table end

Next, we increase the parameter $k$ to be equal to $400$ to retain more features, and presumably, improve performance.
The results are shown in Table \ref{tablemisource400}. We conclude that keeping more features improve the accuracy. However, the way we apply the Mutual Information selection method on source might not be optimal.

%==================Table begin
\begin{table}[ht]
    \begin{center}
    \caption{Accuracy after selecting features based on Mutual Information in Source (k=400) and applying CORAL with  Gaussian Naive Bayes}
    \begin{tabular}[c]{|c|c|c|c|c|}
        \hline
        Source Disaster Event & 500 & 1000 & 2000 & Total \\
        \hline
        2012 Sandy Hurricane & 0.670563117 & 0.725370023 & 0.751186186 & 0.7265859 \\
        2013 Queensland Floods & 0.721959474 & 0.680918495 & 0.632077177 & 0.63512432 \\
        2013 Boston Bombings & 0.538304904 & 0.697602422 & 0.526854118 & 0.720617346 \\
        \hline
    \end{tabular}
    \label{tablemisource400}
   \end{center}
\end{table}
%==================Table end

We combine source and target data disregarding original source labels and assigning label 0 to source samples, and label 1 to target samples. Then we select top K features based on mutual information with the labels.


%==================Table begin
\begin{table}[ht]
    \begin{center}
    \caption{Accuracy after selecting features based on Mutual Information in Source (k=300) combining Source and Target, and applying CORAL with Gaussian Naive Bayes}
    \begin{tabular}[c]{|c|c|c|c|c|}
        \hline
        Source Disaster Event & 500 & 1000 & 2000 & Total \\
        \hline
        2012 Sandy Hurricane & 0.0.725121673   &   0.812701655  & 0.729157619  & 0.824256229 \\
        2013 Queensland Floods & 0.747290799   &   0.510790991  & 0.75946773  &  0.506013627 \\
        2013 Boston Bombings & 0.663243836 &   0.441239214  & 0.29570236   & 0.382660725 \\
        \hline
    \end{tabular}
    \label{tablemisource300Labels}
   \end{center}
\end{table}
%==================Table end


\cleardoublepage

\chapter{Multi-source Domain Adaptation Approach}
\label{mdachapter}

In this chapter, we first define the problem of learning from multiple sources in Section ~\ref{mdaproblemdefinitions}, and then describe the multi-source domain adaptation algorithm ("MDA") proposed in the paper ~\citep{mda} in Section ~\ref{mdaalg}. Finally, we discuss the results obtained after applying MDA on our data in Section ~\ref{mdaexperiments}.


%---------------------------------------------------------------------------------------------------------------------
\section{Problem Definition}
\label{mdaproblemdefinitions}

We define our goal as follows: given tweets from several source domains, train a model to classify tweets from a target domain. The general intuition is that more data should improve performance. Yet, adding more source data, even when expressed via target features, may not necessarily contribute to a higher classification accuracy. In addition, labeled target data, again, is not available. Thus, we explore one of the methods presented in ~\citep{mda}, specificaly, we adopt the idea of modeling the target domain as a linear mixture of the source domains. 

%---------------------------------------------------------------------------------------------------------------------
\section{Multi-source Domain Adaptation Algorithm}
\label{mdaalg}

Notation: \\
$X$ -- features \\
$Y$ -- class labels \\
$P_Y$ -- a distribution of labels / cause \\
$P_{X|Y}$ -- a causal mechanism to generate effect $X$ from cause $Y$ \\
$V_S$ -- a domain-specific selection variable \\
$P_{X|Y, V_S}$ -- a conditional $P_{X|Y}$ in the domain associated with $V_S$

\citet{mda} focus on multi-source domain adaptation from a causal point of view, precisely, they focus on a typical domain adaptation scenario where both $P_Y$ and $P_{X|Y}$ change across domains, but their changes are independent from each other. They assume that the source domains contain rich information such that for each class, $P_{X|Y}^t$ can be approximated by a linear mixture of $P_{X|Y}$ on source domains. 

\citet{mda} discuss several possible domain adaptation situations and their solutions, and we research the case where they model the target as a linear mixture of the sources. 

Consider $P_{X|Y, V_S}$ as the mechanism to generate features from the class labels given the domain. Next, imagine that there exist $L$ elementary "sub-mechanisms", or class conditional feature distributions, $\tilde{P}_{X|Y}^{(l)}$, $l = 1, \cdots, L$, so that the mechanism in each domain, $P_{X|Y, V_S}$, is a mixture of those sub-mechanisms, i.e. $P_{X|Y=c_j, V_S} = \sum_{l=1}^{L} \tilde{a}_{V_S, j, l} \tilde{P}_{X|Y}^{(l)}$, where $\aa_{V_S, j, l}$ depend on both $V_S$ and $j, {\aa}_{V_S, l} \geqslant 0$, and $\sum_{l=1}^{L} \aa_{V_S, j, l} = 1$. 

Consequently, in the multi-source domain adaptation scenario, if for each $j$, the rank of $\left \{P_{X|Y=c_j}^{(i)} | i = 1, \cdots, n \right \}$ is equal to $L$, $P_{X|Y=c_j}^{t}$ can always be represented as a linear mixture of $P_{X|Y=c_j}^{(i)}$, as \citet{mda} state. For each $y$, $P_{X|Y=y}^{t}$ is a mixture of $P_{X|Y=y}$ on the source domains, i.e. there exist $a_{ij}$, which satisfy the constraint $\sum_{i=1}^{n} \aa_{ij} = 1$ for all $j$, such that \[ P_{X|Y=c_j}^{new} = \sum_{i=1}^{n} a_{ij}P_{X|Y=c_j}^{(i)}\] is equal to $P_{X|Y=c_j}^{t}$, where $c_j$ is the $j$th possible value of $Y$. \citet{mda} denote by $P_{Y}^{new}$ a marginal distribution of $Y$, and use $P_{Y}^{new}(c_j)$ as shorthand for $P_{Y}^{new}(Y=c_j)$. The corresponding joint distribution is $P_{X,Y=c_j}^{new}=P_{Y}^{new}(c_j)P_{X|Y=c_j}^{new}$, and the marginal distribution of $X$ is $P_{X}^{new}=\sum_{j=1}^{C}P_{Y}^{new}\sum_{i=1}^{n}a_{ij}P_{X|Y=c_j}^{(i)}$. \citet{mda} aim to match $P_{X}^{new}$ with $P_{X}^{t}$ by tuning the parameters $a_{ij}$ and $P_{Y}^{new}(c_j)$. The constraints are: $P_{Y}^{new}(c_j) \geqslant 0$, and $\sum_{j=1}^{C}P_{Y}^{new}(c_j)=1$. Let $\beta_{ij} \triangleq P_{Y}^{new}(c_{ij})a_{ij}$, which satisfy the condition $\sum_{j=1}^{C}\sum_{i=1}^{n}\beta_{ij} = 1$. When \citet{mda} find the values of $\beta_{ij}$, they reconstruct $p_{Y}^{new}$ and $a_{ij}$ by $P_{Y}^{new}(c_j) = \sum_{i=1}^{n}\beta_{ij}$, and $a_{ij} = \frac{\beta_{ij}}{P_{Y}^{new}(c_j)}$.

%---------------------------------------------------------------------------------------------------------------------
\section{Experiments and Results}
\label{mdaexperiments}

We present the results obtained when no domain adaptation is performed in Table \ref{multisourcenoda}. The source data is expressed via target features and is described in Table \ref{pairstablemulti}. The binary representation (i.e. \textit{0/1}) for both source and target data is used. The source data from different domains is merged together and treated as a whole.

The target data is divided into five folds for cross-validation ~\citep{hastie}. Each fold in turn is used for testing, and the accuracy is recorded in each run. The average results are reported. The number of instances per class in the source data is varied: the results are recorded for source data having $500$ instances per class, $1000$ instances per class, and $2000$ instances per class. All source data is also used as training data. 

As we can see, using more source data generally improves the performance of the classifier.

%==================Table begin
\begin{table}[ht]
    \begin{center}
    \caption{Pairs of Mutli-source--Target Disasters}
    \begin{tabular}[c]{|c|c|c|}
        \hline
        Pair & Source Disaster Event & Target Disaster Event  \\
        \hline
        $SH \rightarrow BB$ & 2012 Sandy Hurricane & 2013 Boston Bombings \\
        $QF \rightarrow BB$ & 2013 Queensland Floods & 2013 Boston Bombings \\

        $SH \rightarrow WT$ & 2012 Sandy Hurricane & 2013 West Texas Explosion \\
        $BB \rightarrow WT$ & 2013 Boston Bombings & 2013 West Texas Explosion \\

        $SH \rightarrow OT$ & 2012 Sandy Hurricane & 2013 Oklahoma Tornado \\
        $QF \rightarrow OT$ & 2013 Queensland Floods & 2013 Oklahoma Tornado \\
        $BB \rightarrow OT$ & 2013 Boston Bombings & 2013 Oklahoma Tornado  \\

        $SH \rightarrow AF$ & 2012 Sandy Hurricane & 2013 Alberta Floods \\
        $QF \rightarrow AF$ & 2013 Queensland Floods & 2013 Alberta Floods \\
        $BB \rightarrow AF$ & 2013 Boston Bombings & 2013 Alberta Floods \\
        \hline
    \end{tabular}
    \label{pairstablemulti}
   \end{center}
\end{table}
%==================Table end


%==================Table begin
\begin{table}[ht]
    \begin{center}
    \caption{Accuracy after running Bernoulli Naive Bayes on multi-source data when no domain adaptation is performed}
    \begin{tabular}[c]{|c|c|c|c|c|c|}
        \hline
        Pair & 500 & 1000 & 2000 & Total \\
        \hline
                             
        $SH, QF \rightarrow BB$ &  0.615428571 & 0.735657143 & 0.596 & 0.688685714  \\ %
                            
        $SH, BB \rightarrow WT$ &  0.850383658	& 0.877821702 & 0.90255107 & 	0.930555488  \\%
                             
        $SH, QF, BB \rightarrow OT$ &  0.82550285	& 0.845725901 & 	0.84960058 & 	0.826957615\\%
                             
        $SH, QF, BB \rightarrow AF$ & 0.72804768 &	0.752038286 & 0.768482834	& 0.775057287 \\ %


        \hline
    \end{tabular}
    \label{multisourcenoda}
   \end{center}
\end{table}
%==================Table end

We first choose a pair of $SH, QF, BB \rightarrow AF$ to experiment with the idea proposed in \citep{mda}. The results are presented in Table \ref{multisourcemda1}. The columns $500$, $1000$, $2000$ and $Total$ mean that $500$, $1000$, $2000$ and all samples per class per source are taken, respectively, and weights are obtained for each of them separately. The weights are presented in Table \ref{sampleweights}, which can be interpreted as follows: the weights for a specific class ($0$ / "Neg" or $1$ / "Pos") across the sources should sum up to $1$. The more one source is similar to the target, the more weight it receives. Also, when the source data is balanced, i.e. $500$, $1000$, $2000$ instances per class per source are used, the weights do not differ much, for example, the weights for the negative class when $500$ intances are used, are almost identical across all the three sources: $0.33779$, $0.33601$ and $0.3262$ for $SH$, $QF$ and $BB$, respectively.


\begin{table}[ht]
    \begin{center}
    \caption{Weights obtained for $SH, QF, BB \rightarrow AF$ (sigma=1.1314)}
    \begin{tabular}[c]{|c|c|c|c|c|c|}

        \hline
        Source & Class & 500 & 1000 & 2000 & Total \\
        \hline
    	\multirow{2}{*}{$SH$} &Neg & 0.33779 & 0.33864 & 0.35149 & 0.38421 \\ & Pos  & 0.34725 & 0.36871 & 0.37865  & 0.38766 \\ 
    	\hline
    	\multirow{2}{*}{$QF$} &Neg  & 0.33601 & 0.34244 & 0.34336 & 0.38544 \\ & Pos  & 0.30849 & 0.25878 & 0.23889  & 0.15128 \\
    	\hline
    	\multirow{2}{*}{$BB$} &Neg  & 0.3262  & 0.31891 & 0.30515 & 0.23035 \\ & Pos  & 0.34426 & 0.37252 & 0.38246  & 0.46105 \\ 

        \hline
    \end{tabular}
    \label{sampleweights}
   \end{center}
\end{table}



The initial value of $sigma$ equal to $1.1314$ is used in the original paper, so we use it as well. However, since the accuracy decreases, we also experiment with different values for $sigma$. The results are presented in Tables \ref{multisourcesigma00001}, \ref{multisourcesigma001}, \ref{multisourcesigma1}, \ref{multisourcesigma10}  for values of sigma set to $0.0001, 0.01, 1, 10, 100$ respectively.

\begin{table}[ht]
    \begin{center}
    \caption{Accuracy after running Gaussian Naive Bayes on multi-source data after applying MDA (sigma=1.1314)}
    \begin{tabular}[c]{|c|c|c|c|c|c|}
        \hline
        Pair & 500 & 1000 & 2000 & Total \\
        \hline
                             
        $SH, QF, BB \rightarrow AF$ & 0.705416920268 &  0.753499695679 & 0.723067559343 & 0.709068776628 \\ %

        \hline
    \end{tabular}
    \label{multisourcemda1}
   \end{center}
\end{table}


%==================Table begin
\begin{table}[ht]
    \begin{center}
    \caption{Accuracy after running Gaussian Naive Bayes on multi-source data after applying MDA (sigma=0.0001)}
    \begin{tabular}[c]{|c|c|c|c|c|c|}
        \hline
        Pair & 500 & 1000 & 2000 & Total \\
        \hline                             
        $SH, QF, BB \rightarrow AF$ & 0.703590992 & 0.723676202  & 0.723676202   & 0.69872185 \\ %
        \hline
    \end{tabular}
    \label{multisourcesigma00001}
   \end{center}
\end{table}
%==================Table end




%==================Table begin
\begin{table}[ht]
    \begin{center}
    \caption{Accuracy after running Gaussian Naive Bayes on multi-source data after applying MDA (sigma=0.01)}
    \begin{tabular}[c]{|c|c|c|c|c|c|}
        \hline
        Pair & 500 & 1000 & 2000 & Total \\
        \hline                             
        $SH, QF, BB \rightarrow AF$ & 0.703590992 & 0.723676202 & 0.723676202 & 0.69872185 \\ %
        \hline
    \end{tabular}
    \label{multisourcesigma001}
   \end{center}
\end{table}
%==================Table end



%==================Table begin
\begin{table}[ht]
    \begin{center}
    \caption{Accuracy after running Gaussian Naive Bayes on multi-source data after applying MDA (sigma=1)}
    \begin{tabular}[c]{|c|c|c|c|c|c|}
        \hline
        Pair & 500 & 1000 & 2000 & Total \\
        \hline                             
        $SH, QF, BB \rightarrow AF$ & 0.703590992 & 0.723676202 & 0.723676202 & 0.69872185 \\ %
        \hline
    \end{tabular}
    \label{multisourcesigma1}
   \end{center}
\end{table}
%==================Table end



%==================Table begin
\begin{table}[ht]
    \begin{center}
    \caption{Accuracy after running Gaussian Naive Bayes on multi-source data after applying MDA (sigma=10)}
    \begin{tabular}[c]{|c|c|c|c|c|c|}
        \hline
        Pair & 500 & 1000 & 2000 & Total \\
        \hline                             
        $SH, QF, BB \rightarrow AF$ & 0.702982349 & 0.721241631 & 0.717589775 & 0.687157638 \\ %
        \hline
    \end{tabular}
    \label{multisourcesigma10}
   \end{center}
\end{table}
%==================Table end



%==================Table begin
\begin{table}[ht]
    \begin{center}
    \caption{Accuracy after running Gaussian Naive Bayes on multi-source data after applying MDA (sigma=100)}
    \begin{tabular}[c]{|c|c|c|c|c|c|}
        \hline
        Pair & 500 & 1000 & 2000 & Total \\
        \hline                             
        $SH, QF, BB \rightarrow AF$ & 0.704808278 & 0.693852708 & 0.661594644 & 0.637857578 \\ %
        \hline
    \end{tabular}
    \label{multisourcesigma100}
   \end{center}
\end{table}
%==================Table end


We decide to further experiment with multi-source domain adaptation. Precisely, motivated by the improved results discussed in Chapter \ref{coralchapter}, we apply the same logic to the multi-source case: we first select features based on Variance Threshold, and then run CORAL. The transformed data is used in training of the Gaussian Naive Bayes classifier. The results are presented in Table \ref{multisourcenvt99coral}. We can see that the accuracy increases across all pairs, e.g. from $0.61$, when no domain adaptation is performed, to $0.81$, after applying feature selection and CORAL, for the pair $SH, QF \rightarrow BB$, when $500$ samples per class are used.


%==================Table begin
\begin{table}[ht]
    \begin{center}
    \caption{Accuracy after running Gaussian Naive Bayes on multi-source data after applying Variance Threshold (0.99) and CORAL}
    \begin{tabular}[c]{|c|c|c|c|c|c|}
        \hline
        Pair & 500 & 1000 & 2000 & Total \\
        \hline
                             
        $SH, QF \rightarrow BB$ & 0.811885714 & 	0.7952	& 0.783542857	& 0.795771429  \\ %
                            
        $SH, BB \rightarrow WT$ & 0.924119594 & 	0.943992949	& 0.94241173 & 	0.932474734   \\%
                             
        $SH, QF, BB \rightarrow OT$ &  0.868976026	& 0.862679013	& 0.820055275	& 0.840034068 \\%
                             
        $SH, QF, BB \rightarrow AF$ & 0.862623332	& 0.866762695	& 0.879185605	& 0.857628755 \\ %


        \hline
    \end{tabular}
    \label{multisourcenvt99coral}
   \end{center}
\end{table}
%==================Table end

In addition, we also confirm that applying CORAL contributes to the accuracy improvement by running an additional set of experiments. Precisely, we do not apply CORAL but we do apply feature selection (Variance Threshold). The results are presented in Table \ref{multisourcenvt99}. We can see that feature selection by itself improves performance, although the accuracy is still lower than when CORAL is applied.


%==================Table begin
\begin{table}[ht]
    \begin{center}
    \caption{Accuracy with Bernoulli Naive Bayes on multi-source data after applying Variance Threshold (0.99)}
    \begin{tabular}[c]{|c|c|c|c|c|c|}
        \hline
        Pair & 500 & 1000 & 2000 & Total \\
        \hline
                             
        $SH, QF \rightarrow BB$ & 0.727428571 &  0.7704  & 0.734285714 & 0.749714286 \\ %
                            
        $SH, BB \rightarrow WT$ &  0.92118289   &  0.928410509  & 0.926264574  & 0.924796731  {}\\%
                             
        $SH, QF, BB \rightarrow OT$ &  0.836763617  &  0.823445032  & 0.82925727   & 0.839428228 \\%
                             
        $SH, QF, BB \rightarrow AF$ & 0.740836072    & 0.741079455  & 0.748629961  & 0.73413774 \\ %


        \hline
    \end{tabular}
    \label{multisourcenvt99}
   \end{center}
\end{table}
%==================Table end
\cleardoublepage

\chapter{Conclusions}
\label{colcusionschapter}

In this chapter, we discuss the overall findings and make conclusions based on the results obtained in all of the experiments.







\cleardoublepage

\chapter{Future Work}
\label{futureworkchapter}

In this chapter, we discuss the possible improvements of the approches used in the work.


% +--------------------------------------------------------------------+
% | Uncomment the lines below to add additional chapters.
% +--------------------------------------------------------------------+

%\cleardoublepage

\chapter{Single-source Domain Adaptation Approach}
\label{coralchapter}

In this chapter, we define the problem of learning from a single source in Section ~\ref{coralproblemdefinitions}, and describe the correlation alignment algorithm ("CORAL") proposed in the paper ~\citep{coral} in Section ~\ref{coralalg}. Then we discuss the results obtained after applying CORAL in Section ~\ref{coralexperiments}.

%---------------------------------------------------------------------------------------------------------------------
\section{Problem Definition}
\label{coralproblemdefinitions}

We define our goal as follows: given tweets from a single source domain, train a model to classify tweets from a target domain. However, direct usage of source data may not give good performance, even if it is expressed via target features. So, we attempt to perform some transformations on source data to align its distribution with the target data distribution, assuming that labels for the target data are not available. As a possible solution, we adopt the method described in ~\citep{coral}, which minimizes the domain shift by aligning the second-order statistics of source and target distributions, without requiring any target labels. 



%---------------------------------------------------------------------------------------------------------------------
\section{Correlation Alignment Algorithm}
\label{coralalg}

~\citep{coral} present an extremely simple domain adaptation method --- CORrelation ALignment (CORAL) --- which works by aligning the distributions of the source and target features in an unsupervised manner. They propose to match the distributions by aligning the second-order statistics, namely, the covariance. More concretely, as ~\citep{coral} states, CORAL aligns the distributions by re-coloring whitened source features with the covariance of the target distribution. CORAL is simple and efficient, as the only computations it needs are:
\begin{itemize}
  \item computing covariance statistics in each domain
  \item applying the whitening and re-coloring linear transformation to the source features.
\end{itemize}

Then, supervised learning proceeds as usual -- training a classifier on the transformed source features.

They describe their method by taking a multi-class classification problem as the running example. Given source-domain training examples $D_S = \left \{ \overrightarrow{x_i} \right \}$, $\overrightarrow{x} \in \mathbb{R}^{D}$ with labels $L_S = \left \{ y_i \right \}$, $ y \in \left \{ {1, \cdots, L} \right \}$, and target data $D_T = \left \{ \overrightarrow{u_i} \right \}$, $\overrightarrow{u} \in \mathbb{R}^{D}$. Here both $\left \{ \overrightarrow{x} \right \}$ and $\left \{ \overrightarrow{u} \right \}$ are the $D$-dimensional feature representations $\varphi(I)$ of input $I$.

Suppose $\mu_s, \mu_t$ and $C_S, C_T$ are the feature vector means and covariance matrices. According to ~\citep{coral}, to minimize the distance between the second-order statistics (covariance) of the source and target features, they apply a linear transformation $A$ to the original source features and use the Frobenius norm as the matrix distance metric: \[ \min_{A} \left \| C_{\hat{S}} - C_T \right \| _{F}^{2} = \min_{A} \left \| A^{T}C_{S}A - C_T \right \| _{F}^{2} \] where $C_S$ is covariance of the transformed source features $D_{s}A$ and $\left \| \cdot   \right \| _{F}^{2}$ denotes the matrix Frobenius norm. Essentially, the solution lies in finding the matrix $A$.

After a series of calculations, which are presented in ~\citep{coral} in detail, the optimal solution can be found as: \[ A^{*} = U_{S}E \\ = (U_{S}\Sigma_{S}^{+\frac{1}{2}}U_{S}^\top)(U_{T[1:r]}\Sigma_{T[1:r]}^{+\frac{1}{2}}U_{T[1:r]}^\top) \] 
This can be interpreted as follows: the first part whitens the source data while the second part re-colors it with the target covariance. 

\textit{Whitening} refers to the process of first de-correlating the data $y$ -- its covariance, $\mathbf{E}(yy^\top)$ is now a diagonal matrix, $\Lambda$. The diagonal elements (eigenvalues) in $\Lambda$ may be the same or different. If we make them all the same, then this is called whitening the data ~\citep{rosalind}. 

\textit{Re-coloring} generally refers to the process of transforming a vector of white random variables into a random vector with a specified covariance matrix ~\citep{miliha}.

As ~\citep{coral} suggests, after CORAL transforms the source features to the target space, a classifier $f_{\overrightarrow{w}}$ parametrized by $\omega$ can be trained on the adjusted source features and directly applied to target features. 

Since correlation alignment changes the features only, it can be applied to any base classifier. In this work, we run experiments using Naive Bayes Classifier. 

The first supervised learning method we use is the multinomial Naive Bayes or multinomial NB model, a probabilistic learning method ~\citep{ir}. It is used for multinomially distributed data, for instance, in text classification where the data are typically represented as word vector counts. We use it with \textit{counts} representation for the data.

An alternative to the multinomial model is the multivariate Bernoulli model or Bernoulli model. It generates an indicator for each term of the vocabulary, either $1$ indicating presence of the term in the document or $0$ indicating absence ~\citep{ir}. We use it with \textit{0/1} representation for the data.

However, since CORAL changes the values of the data from \textit{0/1} to continuous, we need to use Gaussian Naive Bayes algorithm for classification of such data. The likelihood of the features is assumed to be Gaussian: \[ P(x_i|y) = \frac{1}{\sqrt{2\pi \sigma ^{2}_y}} exp\left ( - \frac{(x_i - \mu_y)^2}{2\sigma ^2_y} \right ) \] 

The results of the experiments are discussed in Section ~\ref{coralexperiments}.


%---------------------------------------------------------------------------------------------------------------------
\section{Experiments and Results}
\label{coralexperiments}

We setup the experiments as follows:

\begin{itemize}
  \item use only target features to represent sources
  \item perform 5-fold cross-validation over target and report the average accuracy over the 5 folds (each source is "aligned" with 3 target unlabeled folds using CORAL, 1 labeled target fold is used for testing, 1 labeled target fold is kept for possible use of labeled target data)
  \item vary the number of instances in the sources (i.e. 500 instances per class, 1000 instances per class, etc) - smaller datasets are subsets of th larger datasets
\end{itemize}


%---------------------------------------------------------------------------------------------------------------------
\subsection{Preliminary Results without Domain Adaptation}
\label{preresults}

We present the results obtained when no domain adaptation is performed. The source data is expressed via target features and is described in Table \ref{table1}. Then, two representations are tested: \textit{0/1} and \textit{counts} to determine which representation is more promising i.e. gives better results. Bernoulli Naive Bayes and Multinomial Naive Nayes classifiers are used for \textit{0/1} and \textit{counts} representations respectively. 

The target data is divided into five folds for cross-validation ~\citep{hastie}. Each fold in turn is used for testing, and the accuracy is recorded in each run. The average results are reported. The number of instances per class in the source data is varied: the results are recorded for source data having $500$ instances per class, $1000$ instances per class, and $2000$ instances per class. All source data is also used as training data resulting in slightly better accuracy for $Source 2$ and $Source  3$ ($0.7888$ and $0.7421$, respectively, as compared to $0.7810$ and $0.7346$, respectively, for the number of instances equal to $2000$) in Table \ref{table3}. Generally, balanced training data (i.e. data that has equal number of instances per class) gives better performance.

The results in Table \ref{table3} and Table \ref{table4} indicate that \textit{0/1} representation is more efficient. One of the possible explanations might be that it is unlikely that meaningful words are repeatedly used in a single tweet since tweets are relatively short. Thus, keeping the actual counts does not provide additional relevant information.


%==================Table 1 begin
\begin{table}[ht]
    \begin{center}
    \caption{Target and Source datasets}
    \begin{tabular}[c]{|c|c|c|c|}
        \hline
        Target & Source 1 & Source 2 & Source 3 \\
        \hline
        2013 Alberta Floods & 2012 Sandy Hurricane & 2013 Queensland Floods & 2013 Boston Bombings \\
        \hline
    \end{tabular}
    \label{table1}
   \end{center}
\end{table}
%==================Table 1 end


%==================Table 3
\begin{table}[ht]
    \begin{center}

% +--------------------------------------------------------------------+
% | The table is created with this command
% |
% | \begin{tabular}[pos]{table spec}
% |
% | The "pos" argument specifies the vertical position of the table
% | relative to the baseline of the surrounding text.  Use t, b, or c
% | to specify alignment at the top, bottom, or center.
% |
% | The "table spec" command defines the format of the table
% |   l for a column of left-aligned text
% |   r for a column of right-aligned text
% |   c for centered text
% |   p{width} for a column containing justified text with line breaks
% |   | for a vertical line
% |
% |  In this example, the caption is made to appear above the table
% |  by positioning the \caption command before the \begin{tabular
% |  command. To position the caption below the table, insert the
% |  \caption command after the \end{tabular} command.
% +--------------------------------------------------------------------+

    \caption{Accuracy after running Bernoulli Naive Bayes with 0/1 representation for instances}
    \begin{tabular}[c]{|c|c|c|c|c|}
        \hline
        Sources & 500 & 1000 & 2000 & All \\
        \hline
        Source 1 & 0.699183336 & 0.737669795 & 0.748874085 & 0.714163926 \\
        Source 2 & 0.759834769 & 0.764704356 & 0.781025693 & 0.788819876 \\
        Source 3 & 0.710511727 & 0.716357069 & 0.734625396 & 0.742175131 \\
        \hline
    \end{tabular}

    \label{table3}
   \end{center}
\end{table}
%==================Table 3 end


%==================Table 4 end
\begin{table}[ht]
    \begin{center}
    \caption{Accuracy after running Multinomial Naive Bayes with counts representation for instances}
    \begin{tabular}[c]{|c|c|c|c|c|}
        \hline
        Sources & 500 & 1000 & 2000 & All \\
        \hline
        Source 1 & 0.667883087 & 0.708560882 & 0.727195729 & 0.694068932 \\
        Source 2 & 0.749968752 & 0.758614519 & 0.772012294 & 0.786384119 \\
        Source 3 & 0.676287695 & 0.697966125 & 0.715016795 & 0.720984046 \\
        \hline
    \end{tabular}

    \label{table4}
   \end{center}
\end{table}
%==================Table 4 end

After running the preliminary experiments both with \textit{0/1} representation and \textit{counts} representation using Naive Bayes Classifier ~\citep{tom}, the results obtained in Table \ref{table3} have shown that \textit{0/1} representation gives better performance as compared to the \textit{counts} representation results presented in Table \ref{table4}. Consequently, in the further experiments the \textit{0/1} representation is used.


%==================Table begin
\begin{table}[ht]
    \begin{center}
    \caption{Accuracy after applying CORAL with Gaussian Naive Bayes}
    \begin{tabular}[c]{|c|c|c|c|c|}
        \hline
        Source Disaster Event & 500 & 1000 & 2000 & Total \\
        \hline
        2012 Sandy Hurricane & 0.670077908 & 0.733406627 & 0.77286521 & 0.797223237 \\
        2013 Queensland Floods & 0.666543406 & 0.645719596 & 0.684447955 & 0.657191572\\
        2013 Boston Bombings & 0.643529372 & 0.573255749 & 0.599684114 & 0.649713799 \\
        \hline
    \end{tabular}
    \label{tablecoral}
   \end{center}
\end{table}
%==================Table end

We can see improvement for one pair of source, precisely, 2012 Sandy Hurricane. After applying CORAL, the accuracy increases from $0.74$ to $0.77$ for a subset of $2000$ instances per class and from $0.71$ to $0.79$ for a set that uses all instances. In other cases, the accuracy decreases. One of the possible reasons might be that the initial matrix is very sparse and has a large number of features (precisely, $1334$) as compared to the number of instances ($1000, 2000$ etc). In the original paper ~\citep{coral}, they reduce the original dimentionaly of the dataset to keep top $400$ features based on mutual information ~\citep{hastie}. 

Similarly, we attempt to reduce the dimentionaly of our dataset. We experiment with several dimentionaly reduction techniques. Since we assume that target labeled data is not available, using mutual information for feature selection is not quite applicable.

\subsection{Feature Selection}
\label{varthressub}

%-------------------------------------------------------------------------------------
\subsubsection{Principle Component Analysis}
One of the dimentionality reduction methods that does not require labeled data is Principle Component Analysis (PCA) -- standard linear principal components are obtained from the eigenvectors of the covariance matrix, and give directions in which the data have maximal variance ~\citep{hastie}. Basically, we select $k$ principle components that should describe our data well. One of the recommended methods to choose $k$ is to choose the smallest $k$ for which $99\%$ of variance retained. However, when we choose $k$ in this way, the number of components (i.e. features) retained becomes $1332$, which is only $2$ features less than the original dimentionality of $1334$. Therefore, we decide to not proceed with using PCA on our data.

%-------------------------------------------------------------------------------------
\subsubsection{Variance Threshold}
\label{varthressub}

Feature selector that removes all low-variance features. 

This feature selection algorithm looks only at the features $X$, not the desired outputs $y$, and can thus be used for unsupervised learning ~\citep{varthres}. Specifically, we have a dataset with boolean features, and we want to remove all features that are either one or zero (on or off), for instance, in more than $99\%$ of the samples. Boolean features are Bernoulli random variables, and the variance of such variables is given by: \[ Var[X] = p(1-p)\] so we can select features using the threshold equal to $.99 * (1 - .99)$.

In order to select features, we first concatenate source data and target data, which we call \textit{training target unlabeled -- tTU}. Second, we transform source data and tTU using the extracted features. Then, we run CORAL on source data and tTU. We fit the classifier with transformed source data after the CORAL stage, and we test on \textit{testing target data}.

Features with a training-set variance lower than the specified threshold are removed. We apply feature selection for every combination of source-target fold, so the number of features varies from $160$ to $176$. That is a significant reduction (by $88\%$) compared to the original dimentionaly of the data -- $1334$. 

The value of threshold is varied across the experiments. Precisely, we experiment with $k=0.95, k=0.90, k=0.80$, and we present results in Table \ref{tablevar95}, in Table \ref{tablevar9}, in Table \ref{tablevar8} respectively. The highest accuracy is obtained when the threshold is equal to $0.99$ as described in Table \ref{tablevar99}. 

We decide to further check whether feature selection by itself improves performance, without applying CORAL. We run select features based on Varience Threshold equal to $0.99$ and then run the Bernoulli Naive Bayes classifier. The results presented in Table \ref{tablevar99nocoral} show that applying CORAL does improve performance. Thus, eliminating features with variance lower than $0.99$ and then applying CORAL is more efficient than when CORAL is not applied. 

%==================Table begin
\begin{table}[ht]
    \begin{center}
    \caption{Accuracy after selecting features via Variance Threshold (0.99) without applying CORAL with Bernoulli Naive Bayes}
    \begin{tabular}[c]{|c|c|c|c|c|}
        \hline
        Source Disaster Event & 500 & 1000 & 2000 & Total \\
        \hline
        2012 Sandy Hurricane & 0.669101929 & 0.701010747 & 0.715869636 & 0.651807358 \\
        2013 Queensland Floods & 0.740105627 & 0.740835257 & 0.742905828 & 0.74838643 \\
        2013 Boston Bombings & 0.685786895 & 0.694311377 & 0.69711321 & 0.695408269 \\
        \hline
    \end{tabular}
    \label{tablevar99nocoral}
   \end{center}
\end{table}
%==================Table end


%==================Table begin
\begin{table}[ht]
    \begin{center}
    \caption{Accuracy after selecting features via Variance Threshold (0.99) and applying CORAL with Gaussian Naive Bayes}
    \begin{tabular}[c]{|c|c|c|c|c|}
        \hline
        Source Disaster Event & 500 & 1000 & 2000 & Total \\
        \hline
        2012 Sandy Hurricane & 0.770675727 & 0.843014879 & 0.857263347 & 0.846181082 \\
        2013 Queensland Floods & 0.732553416 & 0.807941416 & 0.813179302 & 0.802460296 \\
        2013 Boston Bombings & 0.712096496 & 0.712092048 & 0.744128673 & 0.791619412 \\
        \hline
    \end{tabular}
    \label{tablevar99}
   \end{center}
\end{table}
%==================Table end



%==================Table begin
\begin{table}[ht]
    \begin{center}
    \caption{Accuracy after selecting features via Variance Threshold (0.95) and applying CORAL with Gaussian Naive Bayes}
    \begin{tabular}[c]{|c|c|c|c|c|}
        \hline
        Source Disaster Event & 500 & 1000 & 2000 & Total \\
        \hline
        2012 Sandy Hurricane & 0.589452614 & 0.775300967 & 0.735233297 & 0.652419188 \\
        2013 Queensland Floods & 0.837409658 & 0.81122223 & 0.837044324 & 0.698949665 \\
        2013 Boston Bombings & 0.639519076 & 0.706117934 & 0.781025248 & 0.734260358 \\
        \hline
    \end{tabular}
    \label{tablevar95}
   \end{center}
\end{table}
%==================Table end

%==================Table begin
\begin{table}[ht]
    \begin{center}
    \caption{Accuracy after selecting features via Variance Threshold (0.90) and applying CORAL with Gaussian Naive Bayes}
    \begin{tabular}[c]{|c|c|c|c|c|}
        \hline
        Source Disaster Event & 500 & 1000 & 2000 & Total \\
        \hline
        2012 Sandy Hurricane & 0.703077612 & 0.681526025 & 0.639758233 & 0.592012176 \\
        2013 Queensland Floods & 0.815486955 & 0.754715498 & 0.734633995 & 0.657777097 \\
        2013 Boston Bombings & 0.743512617 & 0.700768106 & 0.679461385 & 0.626111366 \\
        \hline
    \end{tabular}
    \label{tablevar9}
   \end{center}
\end{table}
%==================Table end

%==================Table begin
\begin{table}[ht]
    \begin{center}
    \caption{Accuracy after selecting features via Variance Threshold (0.80) and applying CORAL with Gaussian Naive Bayes}
    \begin{tabular}[c]{|c|c|c|c|c|}
        \hline
        Source Disaster Event & 500 & 1000 & 2000 & Total \\
        \hline
        2012 Sandy Hurricane & 0.561565954 & 0.590795039 & 0.585071351 & 0.517477091 \\
        2013 Queensland Floods & 0.62684663 & 0.616980465 & 0.636341457 & 0.644623001 \\
        2013 Boston Bombings & 0.743512617 & 0.700768106 & 0.679461385 & 0.626111366 \\
        \hline
    \end{tabular}
    \label{tablevar8}
   \end{center}
\end{table}
%==================Table end

We decide to test the method further on all pairs of source and target disasters described in Table ~\ref{tableadddata}. The results of applying feature selection followed by CORAL are shown in Table ~\ref{tablevar99adddata}, and the results of applying feature selection without CORAL are presented in Table ~\ref{tablevar99adddatanocoral}. 

In addition, we present the number of features retained after applying Variance Threshold to the pair 2013 Boston Bombings -- 2013 Alberta Floods when the different amount of the source instances is used in Figure \ref{featuresretained99figure}. Precisely, $1000$ number of the source instances means that there are $500$ instances of the class \textit{1}, and $500$ instances of the class \textit{0}, the same logic applies to $2000$ and $4000$. $8750$ instances means that we use all the instances in the source data, and it becomes imbalanced. The plot gives a better understanding of the effect of Variance Threshold on reducing the feature space. The number of features for other pairs is in a similar range of $159-187$, which varies for different pairs and their folds.

Interestingly, when the threshold is set to $0.95$, the number of features retained reduces even further, and is presented in Figure \ref{featuresretained95figure}. 


%==================Table begin
\begin{table}[ht]
    \begin{center}
    \caption{All Pairs of Source-Target Disasters}
    \begin{tabular}[c]{|c|c|c|}
        \hline
        Pair & Source Disaster Event & Target Disaster Event  \\
        \hline
        $SH \rightarrow QF$ & 2012 Sandy Hurricane & 2013 Queensland Floods \\

        $SH \rightarrow BB$ & 2012 Sandy Hurricane & 2013 Boston Bombings \\
        $QF \rightarrow BB$ & 2013 Queensland Floods & 2013 Boston Bombings \\

        $SH \rightarrow WT$ & 2012 Sandy Hurricane & 2013 West Texas Explosion \\
        $BB \rightarrow WT$ & 2013 Boston Bombings & 2013 West Texas Explosion \\

        $SH \rightarrow OT$ & 2012 Sandy Hurricane & 2013 Oklahoma Tornado \\
        $QF \rightarrow OT$ & 2013 Queensland Floods & 2013 Oklahoma Tornado \\
        $BB \rightarrow OT$ & 2013 Boston Bombings & 2013 Oklahoma Tornado  \\

        $SH \rightarrow AF$ & 2012 Sandy Hurricane & 2013 Alberta Floods \\
        $QF \rightarrow AF$ & 2013 Queensland Floods & 2013 Alberta Floods \\
        $BB \rightarrow AF$ & 2013 Boston Bombings & 2013 Alberta Floods \\
        \hline
    \end{tabular}
    \label{tableadddata}
   \end{center}
\end{table}
%==================Table end

%==================Table begin
\begin{table}[ht]
    \begin{center}
    \caption{Accuracy after selecting features via Variance Threshold (0.99) without applying CORAL with Bernoulli Naive Bayes. All pairs}
    \begin{tabular}[c]{|c|c|c|c|c|c|}
        \hline
        Pair & 500 & 1000 & 2000 & Total \\
        \hline
        $SH \rightarrow QF$ & 0.645647344  &    0.76355232  &  0.773312788  & 0.724891438 \\ %

        $SH \rightarrow BB$ & 0.694285714   &  0.7664  &  0.703314286  & 0.6864 \\ %
        $QF \rightarrow BB$ & 0.712685714   &  0.703314286  & 0.716342857  & 0.717485714 \\ %

        $SH \rightarrow WT$ & 0.682135879  &   0.735997122  & 0.714882615  & 0.738481467 \\ %
        $BB \rightarrow WT$ & 0.923893032   &  0.931683902  & 0.93157129   & 0.94241173 \\ %

        $SH \rightarrow OT$ & 0.763017791   &  0.768710137  & 0.795108695  & 0.762413197 \\ %
        $QF \rightarrow OT$ & 0.796802144   &  0.801283322  & 0.815331232  & 0.815209727 \\ %
        $BB \rightarrow OT$ & 0.796561919   &  0.791354935  & 0.806612765  & 0.808791942 \\ %

        $SH \rightarrow AF$ & 0.669101929 & 0.701010747 & 0.715869636 & 0.651807358 \\ %
        $QF \rightarrow AF$ & 0.740105627 & 0.740835257 & 0.742905828 & 0.74838643 \\  %
        $BB \rightarrow AF$ & 0.685786895 & 0.694311377 & 0.69711321 & 0.695408269 \\ %


        \hline
    \end{tabular}
    \label{tablevar99adddatanocoral}
   \end{center}
\end{table}
%==================Table end

%==================Table begin
\begin{table}[ht]
    \begin{center}
    \caption{Accuracy after selecting features via Variance Threshold (0.99) and applying CORAL with Gaussian Naive Bayes. All pairs}
    \begin{tabular}[c]{|c|c|c|c|c|c|}
        \hline
        Pair & 500 & 1000 & 2000 & Total \\
        \hline
        $SH \rightarrow QF$ & 0.751221161 & 0.851656398 & 0.852042661 & 0.838684627 \\

        $SH \rightarrow BB$ & 0.789714286 & 0.764114286 & 0.827314286 & 0.766971429 \\
        $QF \rightarrow BB$ & 0.834057143 & 0.819657143 & 0.804571429 & 0.68 \\

        $SH \rightarrow WT$ & 0.802055629 & 0.738821915 & 0.674232334 & 0.836269315 \\
        $BB \rightarrow WT$ & 0.888098892 & 0.945347159 & 0.94568627 & 0.949412532 \\

        $SH \rightarrow OT$ & 0.85359559 & 0.858683049 & 0.857835445 & 0.753334467\\
        $QF \rightarrow OT$ & 0.82743645 & 0.867645187 & 0.873215221 & 0.815452151 \\
        $BB \rightarrow OT$ & 0.792924312 & 0.853962013 & 0.827806759 & 0.823568371 \\

        $SH \rightarrow AF$ & 0.770675727 & 0.843014879 & 0.857263347 & 0.846181082 \\
        $QF \rightarrow AF$ & 0.732553416 & 0.807941416 & 0.813179302 & 0.802460296 \\
        $BB \rightarrow AF$ & 0.712096496 & 0.712092048 & 0.744128673 & 0.791619412 \\

        \hline
    \end{tabular}
    \label{tablevar99adddata}
   \end{center}
\end{table}
%==================Table end

%==================Figure begin
\begin{figure}[ht]
\centering
\caption{Number of features retained after applying Variance Threshold (0.99)}
\begin{tikzpicture}
\begin{axis}[
    title={Source -- 2013 Boston Bombings, Target -- 2013 Alberta Floods},
    xlabel={Number of source instances},
    ylabel={Number of features retained},
    width=0.8\textwidth,
    % xmin=900, xmax=10000,
    % ymin=150, ymax=190,
    xtick={1000, 2000, 4000, 8750},
    % ytick={155, 165, 175, 185},
    legend pos=north west,
    ymajorgrids=true,
    grid style=dashed,
]
 
\addplot[only marks,
    color=blue,
    mark = *
    %mark=square,
    ]
    table {vt_features_99.dat};
 
\end{axis}
\end{tikzpicture}
\label{featuresretained99figure}
\end{figure}
%==================Figure end


%==================Figure begin
\begin{figure}[ht]
\centering
\caption{Number of features retained after applying Variance Threshold (0.95)}
\begin{tikzpicture}
\begin{axis}[
    title={Source -- 2013 Boston Bombings, Target -- 2013 Alberta Floods},
    xlabel={Number of source instances},
    ylabel={Number of features retained},
    width=0.8\textwidth,
    % xmin=900, xmax=10000,
    % ymin=150, ymax=190,
    xtick={1000, 2000, 4000, 8750},
    % ytick={155, 165, 175, 185},
    legend pos=north west,
    ymajorgrids=true,
    grid style=dashed,
]
 
\addplot[only marks,
    color=blue,
    mark = *
    %mark=square,
    ]
    table {vt_features_95.dat};
 
\end{axis}
\end{tikzpicture}
\label{featuresretained95figure}
\end{figure}
%==================Figure end


%==================Figure begin
\begin{figure}[ht]
\centering
\caption{Number of features retained after applying Variance Threshold (0.90)}
\begin{tikzpicture}
\begin{axis}[
    title={Source -- 2013 Boston Bombings, Target -- 2013 Alberta Floods},
    xlabel={Number of source instances},
    ylabel={Number of features retained},
    width=0.8\textwidth,
    % xmin=900, xmax=10000,
    % ymin=150, ymax=190,
    xtick={1000, 2000, 4000, 8750},
    % ytick={155, 165, 175, 185},
    legend pos=north west,
    ymajorgrids=true,
    grid style=dashed,
]
 
\addplot[only marks,
    color=blue,
    mark = *
    %mark=square,
    ]
    table {vt_features_90.dat};
 
\end{axis}
\end{tikzpicture}
\label{featuresretained90figure}
\end{figure}
%==================Figure end


We decide to further experiment with this feature selection method, precisely, we now rank features by their variance i.e. calcucating $p(1-p)$ for each feature across all samples, in the descending order. We then take top $k$ features and run the experiments again. The results for $k = 190$ without applying CORAl and with applying CORAL are shown in Table \ref{top190vartablenocoral} and in Table \ref{top190vartablecoral}, respectively, and the results for $k = 170$ with applying CORAL are shown in Table \ref{top170vartablecoral}.

%==================Table begin
\begin{table}[ht]
    \begin{center}
    \caption{Accuracy after ranking features by variance in the descending order, and selecting top k features (k=190), without applying CORAL with Bernoulli Naive Bayes. All pairs}
    \begin{tabular}[c]{|c|c|c|c|c|c|}
        \hline
        Pair & 500 & 1000 & 2000 & Total \\
        \hline
        $SH \rightarrow QF$ &  0.650399414   & 0.77138584   & 0.78179063   & 0.730799495 \\

        $SH \rightarrow BB$ &  0.693485714   & 0.766171429  & 0.7048   & 0.692342857 \\
        $QF \rightarrow BB$ &  0.709714286   & 0.710057143  & 0.718628571  & 0.725714286 \\

        $SH \rightarrow WT$ &  0.681909635  & 0.735659031 & 0.7162381 & 0.73915905 \\
        $BB \rightarrow WT$ &  0.924231696   & 0.931345237  & 0.932022884  & 0.942976255 \\

        $SH \rightarrow OT$ &  0.766046699   & 0.775854974  & 0.804795391  & 0.778155509 \\
        $QF \rightarrow OT$ &  0.801041925   & 0.804068706  & 0.818116396  & 0.823080443 \\
        $BB \rightarrow OT$ &  0.798257348   & 0.790748508  & 0.807339011  & 0.81714736 \\

        $SH \rightarrow AF$ &  0.67555547    & 0.710753775  & 0.718305616  & 0.658506134 \\
        $QF \rightarrow AF$ &  0.744489856   & 0.742174641  & 0.746437142  & 0.752405103\\
        $BB \rightarrow AF$ &  0.694069181   & 0.70393327   & 0.702228626  & 0.693703105 \\

        \hline
    \end{tabular}
    \label{top190vartablenocoral}
   \end{center}
\end{table}
%==================Table end


%==================Table begin
\begin{table}[ht]
    \begin{center}
    \caption{Accuracy after ranking features by variance in the descending order, selecting top k features (k=190), and applying CORAL with Gaussian Naive Bayes. All pairs}
    \begin{tabular}[c]{|c|c|c|c|c|c|}
        \hline
        Pair & 500 & 1000 & 2000 & Total \\
        \hline
        $SH \rightarrow QF$ & 0.78114606  & 0.858591907 & 0.860391309 & 0.830336061    \\ %
                              
        $SH \rightarrow BB$ & 0.793714286 & 0.7704      & 0.818742857 & 0.766857143    \\ %
        $QF \rightarrow BB$ & 0.753828571 & 0.791085714 & 0.810285714 & 0.670628571    \\%
                               
        $SH \rightarrow WT$ & 0.791216081 & 0.741531547 & 0.686653802 & 0.832543181     \\%
        $BB \rightarrow WT$ & 0.883921163 & 0.945686079 & 0.948508832 & 0.95076769     \\%
                               
        $SH \rightarrow OT$ & 0.850326605 & 0.866917695 & 0.860378551 & 0.770045156    \\%
        $QF \rightarrow OT$ & 0.806971048 & 0.86643248  & 0.867402763 & 0.827076627    \\%
        $BB \rightarrow OT$ & 0.792201513 & 0.851418247 & 0.81460484  & 0.827200625     \\%
                               
        $SH \rightarrow AF$ & 0.771767948 & 0.834732001 & 0.863352517 & 0.853122574  \\ %
        $QF \rightarrow AF$ & 0.722325697 & 0.806237216 & 0.816832789 & 0.759592573 \\  %
        $BB \rightarrow AF$ & 0.652545809 & 0.705151075 & 0.70296389  & 0.771159157  \\%

        \hline
    \end{tabular}
    \label{top190vartablecoral}
   \end{center}
\end{table}
%==================Table end

%==================Table begin
\begin{table}[ht]
    \begin{center}
    \caption{Accuracy after ranking features by variance in the descending order, selecting top k features (k=170), and applying CORAL with Gaussian Naive Bayes. All pairs}
    \begin{tabular}[c]{|c|c|c|c|c|c|}
        \hline
        Pair & 500 & 1000 & 2000 & Total \\
        \hline
        $SH \rightarrow QF$ & 0.744414351 & 0.842666149 & 0.854740156   & 0.83958338   \\ %
                             
        $SH \rightarrow BB$ & 0.789714286 & 0.763085714 & 0.827428571   & 0.765942857   \\ %
        $QF \rightarrow BB$ & 0.812914286 & 0.8128  & 0.808457143       & 0.667657143   \\%
                             
        $SH \rightarrow WT$ & 0.80826528  & 0.739837334 & 0.676491072 & 0.834688224    \\%
        $BB \rightarrow WT$ & 0.886856528 & 0.944217854 & 0.945460409 & 0.949525462   \\%
                             
        $SH \rightarrow OT$ & 0.852627361 & 0.858683049 & 0.858319706 & 0.75309219   \\%
        $QF \rightarrow OT$ & 0.835914399 & 0.866071557 & 0.868976759 & 0.823080516   \\%
        $BB \rightarrow OT$ & 0.793046184 & 0.84390794  & 0.824658106 & 0.827927604    \\%
                             
        $SH \rightarrow AF$ & 0.76263675  & 0.8427712   & 0.858968584 & 0.845937625 \\ %
        $QF \rightarrow AF$ & 0.73230981  & 0.807819613 & 0.814884317 & 0.791254449 \\  %
        $BB \rightarrow AF$ & 0.718186778 & 0.711848591 & 0.738403725 & 0.788696593 \\%

        \hline
    \end{tabular}
    \label{top170vartablecoral}
   \end{center}
\end{table}
%==================Table end

%-------------------------------------------------------------------------------------
\subsubsection{Truncated SVD aka Latent Semantic Analysis}

This transformer performs linear dimensionality reduction by means of truncated singular value decomposition (SVD) ~\citep{ir}. Contrary to PCA, this estimator does not center the data before computing the singular value decomposition. This means it can work with scipy.sparse matrices efficiently. 

The results on our data show that Truncated SVD does not contribute to better accuracy. We manually choose the number of components $k$. Concretely, we experiment with $k=677$, choosing this values as $50\%$ of the original number of features ($1334$), with $k=400$ and with $k=170$, choosing this value as an average number of features obtained after Variance Threshold discussed in \ref{varthressub} that give the best performance so far.

%==================Table begin
\begin{table}[!h]%[ht]
    \begin{center}
    \caption{Accuracy after selecting features via Truncated SVD (k=677) and applying CORAL with Gaussian Naive Bayes}
    \begin{tabular}[c]{|c|c|c|c|c|}
        \hline
        Source Disaster Event & 500 & 1000 & 2000 & Total \\
        \hline
        2012 Sandy Hurricane & 0.610885735 & 0.61076534 & 0.631833052 & 0.614783494 \\
        2013 Queensland Floods & 0.644013172 & 0.648520761 & 0.651077431 & 0.646450264 \\
        2013 Boston Bombings & 0.635488616 & 0.58519041 & 0.606746297 & 0.609668153 \\
        \hline
    \end{tabular}
    \label{tabletrunsvd677}
   \end{center}
\end{table}
%==================Table end

%==================Table begin
\begin{table}[!h]%[ht]
    \begin{center}
    \caption{Accuracy after selecting features via Truncated SVD (k=400) and applying CORAL with Gaussian Naive Bayes}
    \begin{tabular}[c]{|c|c|c|c|c|}
        \hline
        Source Disaster Event & 500 & 1000 & 2000 & Total \\
        \hline
        2012 Sandy Hurricane & 0.629033148 & 0.630251842 & 0.64462226 & 0.634392095 \\
        2013 Queensland Floods & 0.656436971 & 0.617708835 & 0.626234503 & 0.606870175 \\
        2013 Boston Bombings & 0.646330092 & 0.56668048 & 0.578736203 & 0.582876604 \\
        \hline
    \end{tabular}
    \label{tabletrunsvd400}
   \end{center}
\end{table}
%==================Table end

%==================Table begin
\begin{table}[!h]%[ht]
    \begin{center}
    \caption{Accuracy after selecting features via Truncated SVD (k=170) and applying CORAL with Gaussian Naive Bayes}
    \begin{tabular}[c]{|c|c|c|c|c|}
        \hline
        Source Disaster Event & 500 & 1000 & 2000 & Total \\
        \hline
        2012 Sandy Hurricane & 0.610885735 & 0.61076534 & 0.631833052 & 0.614783494 \\
        2013 Queensland Floods & 0.644013172 & 0.648520761 & 0.651077431 & 0.646450264 \\
        2013 Boston Bombings & 0.635488616 & 0.58519041 & 0.606746297 & 0.609668153 \\
        \hline
    \end{tabular}
    \label{tabletrunsvd170}
   \end{center}
\end{table}
%==================Table end


%-------------------------------------------------------------------------------------
\subsubsection{Mutual Information}

The mutual information of two random variables is a natural measure of dependence between the two variables ~\citep{hastie}, which can be expressed as follows: \[I(x,y) = \sum_{x,y} P(x,y) \ln {{P(x,y)}\over{P(x) P(y)}} \]

Our first experiment in this subsection we setup as follows:
  \begin{itemize}
  \item express source via target features
  \item select top K features from source based on Mutual Information
  \item transform the source and target validation fold to be expressed via newly obtained top $K$ features 
  \item run CORAL on transformed source and transformed target obtained at the previous step
  \item fit the classifier with the transformed source obtained at the previous step
  \item transform the target test fold to be expressed via top $K$ features
  \item test the classifier on target test fold
  \end{itemize}

We select $k=300$ and present the results in Table \ref{tablemisource300}. 

%==================Table begin
\begin{table}[ht]
    \begin{center}
    \caption{Accuracy after selecting features based on Mutual Information in Source (k=300) and applying CORAL with Gaussian Naive Bayes}
    \begin{tabular}[c]{|c|c|c|c|c|}
        \hline
        Source Disaster Event & 500 & 1000 & 2000 & Total \\
        \hline
        2012 Sandy Hurricane & 0.668858769 & 0.710877135 & 0.648762884 & 0.698939879 \\
        2013 Queensland Floods & 0.704055592 & 0.742418098 & 0.708438635 & 0.63378575 \\
        2013 Boston Bombings & 0.46681585 & 0.6857886 & 0.500670619 & 0.690170383 \\
        \hline
    \end{tabular}
    \label{tablemisource300}
   \end{center}
\end{table}
%==================Table end

Next, we increase the parameter $k$ to be equal to $400$ to retain more features, and presumably, improve performance.
The results are shown in Table \ref{tablemisource400}. We conclude that keeping more features improve the accuracy. However, the way we apply the Mutual Information selection method on source might not be optimal.

%==================Table begin
\begin{table}[ht]
    \begin{center}
    \caption{Accuracy after selecting features based on Mutual Information in Source (k=400) and applying CORAL with  Gaussian Naive Bayes}
    \begin{tabular}[c]{|c|c|c|c|c|}
        \hline
        Source Disaster Event & 500 & 1000 & 2000 & Total \\
        \hline
        2012 Sandy Hurricane & 0.670563117 & 0.725370023 & 0.751186186 & 0.7265859 \\
        2013 Queensland Floods & 0.721959474 & 0.680918495 & 0.632077177 & 0.63512432 \\
        2013 Boston Bombings & 0.538304904 & 0.697602422 & 0.526854118 & 0.720617346 \\
        \hline
    \end{tabular}
    \label{tablemisource400}
   \end{center}
\end{table}
%==================Table end

We combine source and target data disregarding original source labels and assigning label 0 to source samples, and label 1 to target samples. Then we select top K features based on mutual information with the labels.


%==================Table begin
\begin{table}[ht]
    \begin{center}
    \caption{Accuracy after selecting features based on Mutual Information in Source (k=300) combining Source and Target, and applying CORAL with Gaussian Naive Bayes}
    \begin{tabular}[c]{|c|c|c|c|c|}
        \hline
        Source Disaster Event & 500 & 1000 & 2000 & Total \\
        \hline
        2012 Sandy Hurricane & 0.0.725121673   &   0.812701655  & 0.729157619  & 0.824256229 \\
        2013 Queensland Floods & 0.747290799   &   0.510790991  & 0.75946773  &  0.506013627 \\
        2013 Boston Bombings & 0.663243836 &   0.441239214  & 0.29570236   & 0.382660725 \\
        \hline
    \end{tabular}
    \label{tablemisource300Labels}
   \end{center}
\end{table}
%==================Table end


%\cleardoublepage

\chapter{Multi-source Domain Adaptation Approach}
\label{mdachapter}

In this chapter, we first define the problem of learning from multiple sources in Section ~\ref{mdaproblemdefinitions}, and then describe the multi-source domain adaptation algorithm ("MDA") proposed in the paper ~\citep{mda} in Section ~\ref{mdaalg}. Finally, we discuss the results obtained after applying MDA on our data in Section ~\ref{mdaexperiments}.


%---------------------------------------------------------------------------------------------------------------------
\section{Problem Definition}
\label{mdaproblemdefinitions}

We define our goal as follows: given tweets from several source domains, train a model to classify tweets from a target domain. The general intuition is that more data should improve performance. Yet, adding more source data, even when expressed via target features, may not necessarily contribute to a higher classification accuracy. In addition, labeled target data, again, is not available. Thus, we explore one of the methods presented in ~\citep{mda}, specificaly, we adopt the idea of modeling the target domain as a linear mixture of the source domains. 

%---------------------------------------------------------------------------------------------------------------------
\section{Multi-source Domain Adaptation Algorithm}
\label{mdaalg}

Notation: \\
$X$ -- features \\
$Y$ -- class labels \\
$P_Y$ -- a distribution of labels / cause \\
$P_{X|Y}$ -- a causal mechanism to generate effect $X$ from cause $Y$ \\
$V_S$ -- a domain-specific selection variable \\
$P_{X|Y, V_S}$ -- a conditional $P_{X|Y}$ in the domain associated with $V_S$

\citet{mda} focus on multi-source domain adaptation from a causal point of view, precisely, they focus on a typical domain adaptation scenario where both $P_Y$ and $P_{X|Y}$ change across domains, but their changes are independent from each other. They assume that the source domains contain rich information such that for each class, $P_{X|Y}^t$ can be approximated by a linear mixture of $P_{X|Y}$ on source domains. 

\citet{mda} discuss several possible domain adaptation situations and their solutions, and we research the case where they model the target as a linear mixture of the sources. 

Consider $P_{X|Y, V_S}$ as the mechanism to generate features from the class labels given the domain. Next, imagine that there exist $L$ elementary "sub-mechanisms", or class conditional feature distributions, $\tilde{P}_{X|Y}^{(l)}$, $l = 1, \cdots, L$, so that the mechanism in each domain, $P_{X|Y, V_S}$, is a mixture of those sub-mechanisms, i.e. $P_{X|Y=c_j, V_S} = \sum_{l=1}^{L} \tilde{a}_{V_S, j, l} \tilde{P}_{X|Y}^{(l)}$, where $\aa_{V_S, j, l}$ depend on both $V_S$ and $j, {\aa}_{V_S, l} \geqslant 0$, and $\sum_{l=1}^{L} \aa_{V_S, j, l} = 1$. 

Consequently, in the multi-source domain adaptation scenario, if for each $j$, the rank of $\left \{P_{X|Y=c_j}^{(i)} | i = 1, \cdots, n \right \}$ is equal to $L$, $P_{X|Y=c_j}^{t}$ can always be represented as a linear mixture of $P_{X|Y=c_j}^{(i)}$, as \citet{mda} state. For each $y$, $P_{X|Y=y}^{t}$ is a mixture of $P_{X|Y=y}$ on the source domains, i.e. there exist $a_{ij}$, which satisfy the constraint $\sum_{i=1}^{n} \aa_{ij} = 1$ for all $j$, such that \[ P_{X|Y=c_j}^{new} = \sum_{i=1}^{n} a_{ij}P_{X|Y=c_j}^{(i)}\] is equal to $P_{X|Y=c_j}^{t}$, where $c_j$ is the $j$th possible value of $Y$. \citet{mda} denote by $P_{Y}^{new}$ a marginal distribution of $Y$, and use $P_{Y}^{new}(c_j)$ as shorthand for $P_{Y}^{new}(Y=c_j)$. The corresponding joint distribution is $P_{X,Y=c_j}^{new}=P_{Y}^{new}(c_j)P_{X|Y=c_j}^{new}$, and the marginal distribution of $X$ is $P_{X}^{new}=\sum_{j=1}^{C}P_{Y}^{new}\sum_{i=1}^{n}a_{ij}P_{X|Y=c_j}^{(i)}$. \citet{mda} aim to match $P_{X}^{new}$ with $P_{X}^{t}$ by tuning the parameters $a_{ij}$ and $P_{Y}^{new}(c_j)$. The constraints are: $P_{Y}^{new}(c_j) \geqslant 0$, and $\sum_{j=1}^{C}P_{Y}^{new}(c_j)=1$. Let $\beta_{ij} \triangleq P_{Y}^{new}(c_{ij})a_{ij}$, which satisfy the condition $\sum_{j=1}^{C}\sum_{i=1}^{n}\beta_{ij} = 1$. When \citet{mda} find the values of $\beta_{ij}$, they reconstruct $p_{Y}^{new}$ and $a_{ij}$ by $P_{Y}^{new}(c_j) = \sum_{i=1}^{n}\beta_{ij}$, and $a_{ij} = \frac{\beta_{ij}}{P_{Y}^{new}(c_j)}$.

%---------------------------------------------------------------------------------------------------------------------
\section{Experiments and Results}
\label{mdaexperiments}

We present the results obtained when no domain adaptation is performed in Table \ref{multisourcenoda}. The source data is expressed via target features and is described in Table \ref{pairstablemulti}. The binary representation (i.e. \textit{0/1}) for both source and target data is used. The source data from different domains is merged together and treated as a whole.

The target data is divided into five folds for cross-validation ~\citep{hastie}. Each fold in turn is used for testing, and the accuracy is recorded in each run. The average results are reported. The number of instances per class in the source data is varied: the results are recorded for source data having $500$ instances per class, $1000$ instances per class, and $2000$ instances per class. All source data is also used as training data. 

As we can see, using more source data generally improves the performance of the classifier.

%==================Table begin
\begin{table}[ht]
    \begin{center}
    \caption{Pairs of Mutli-source--Target Disasters}
    \begin{tabular}[c]{|c|c|c|}
        \hline
        Pair & Source Disaster Event & Target Disaster Event  \\
        \hline
        $SH \rightarrow BB$ & 2012 Sandy Hurricane & 2013 Boston Bombings \\
        $QF \rightarrow BB$ & 2013 Queensland Floods & 2013 Boston Bombings \\

        $SH \rightarrow WT$ & 2012 Sandy Hurricane & 2013 West Texas Explosion \\
        $BB \rightarrow WT$ & 2013 Boston Bombings & 2013 West Texas Explosion \\

        $SH \rightarrow OT$ & 2012 Sandy Hurricane & 2013 Oklahoma Tornado \\
        $QF \rightarrow OT$ & 2013 Queensland Floods & 2013 Oklahoma Tornado \\
        $BB \rightarrow OT$ & 2013 Boston Bombings & 2013 Oklahoma Tornado  \\

        $SH \rightarrow AF$ & 2012 Sandy Hurricane & 2013 Alberta Floods \\
        $QF \rightarrow AF$ & 2013 Queensland Floods & 2013 Alberta Floods \\
        $BB \rightarrow AF$ & 2013 Boston Bombings & 2013 Alberta Floods \\
        \hline
    \end{tabular}
    \label{pairstablemulti}
   \end{center}
\end{table}
%==================Table end


%==================Table begin
\begin{table}[ht]
    \begin{center}
    \caption{Accuracy after running Bernoulli Naive Bayes on multi-source data when no domain adaptation is performed}
    \begin{tabular}[c]{|c|c|c|c|c|c|}
        \hline
        Pair & 500 & 1000 & 2000 & Total \\
        \hline
                             
        $SH, QF \rightarrow BB$ &  0.615428571 & 0.735657143 & 0.596 & 0.688685714  \\ %
                            
        $SH, BB \rightarrow WT$ &  0.850383658	& 0.877821702 & 0.90255107 & 	0.930555488  \\%
                             
        $SH, QF, BB \rightarrow OT$ &  0.82550285	& 0.845725901 & 	0.84960058 & 	0.826957615\\%
                             
        $SH, QF, BB \rightarrow AF$ & 0.72804768 &	0.752038286 & 0.768482834	& 0.775057287 \\ %


        \hline
    \end{tabular}
    \label{multisourcenoda}
   \end{center}
\end{table}
%==================Table end

We first choose a pair of $SH, QF, BB \rightarrow AF$ to experiment with the idea proposed in \citep{mda}. The results are presented in Table \ref{multisourcemda1}. The columns $500$, $1000$, $2000$ and $Total$ mean that $500$, $1000$, $2000$ and all samples per class per source are taken, respectively, and weights are obtained for each of them separately. The weights are presented in Table \ref{sampleweights}, which can be interpreted as follows: the weights for a specific class ($0$ / "Neg" or $1$ / "Pos") across the sources should sum up to $1$. The more one source is similar to the target, the more weight it receives. Also, when the source data is balanced, i.e. $500$, $1000$, $2000$ instances per class per source are used, the weights do not differ much, for example, the weights for the negative class when $500$ intances are used, are almost identical across all the three sources: $0.33779$, $0.33601$ and $0.3262$ for $SH$, $QF$ and $BB$, respectively.


\begin{table}[ht]
    \begin{center}
    \caption{Weights obtained for $SH, QF, BB \rightarrow AF$ (sigma=1.1314)}
    \begin{tabular}[c]{|c|c|c|c|c|c|}

        \hline
        Source & Class & 500 & 1000 & 2000 & Total \\
        \hline
    	\multirow{2}{*}{$SH$} &Neg & 0.33779 & 0.33864 & 0.35149 & 0.38421 \\ & Pos  & 0.34725 & 0.36871 & 0.37865  & 0.38766 \\ 
    	\hline
    	\multirow{2}{*}{$QF$} &Neg  & 0.33601 & 0.34244 & 0.34336 & 0.38544 \\ & Pos  & 0.30849 & 0.25878 & 0.23889  & 0.15128 \\
    	\hline
    	\multirow{2}{*}{$BB$} &Neg  & 0.3262  & 0.31891 & 0.30515 & 0.23035 \\ & Pos  & 0.34426 & 0.37252 & 0.38246  & 0.46105 \\ 

        \hline
    \end{tabular}
    \label{sampleweights}
   \end{center}
\end{table}



The initial value of $sigma$ equal to $1.1314$ is used in the original paper, so we use it as well. However, since the accuracy decreases, we also experiment with different values for $sigma$. The results are presented in Tables \ref{multisourcesigma00001}, \ref{multisourcesigma001}, \ref{multisourcesigma1}, \ref{multisourcesigma10}  for values of sigma set to $0.0001, 0.01, 1, 10, 100$ respectively.

\begin{table}[ht]
    \begin{center}
    \caption{Accuracy after running Gaussian Naive Bayes on multi-source data after applying MDA (sigma=1.1314)}
    \begin{tabular}[c]{|c|c|c|c|c|c|}
        \hline
        Pair & 500 & 1000 & 2000 & Total \\
        \hline
                             
        $SH, QF, BB \rightarrow AF$ & 0.705416920268 &  0.753499695679 & 0.723067559343 & 0.709068776628 \\ %

        \hline
    \end{tabular}
    \label{multisourcemda1}
   \end{center}
\end{table}


%==================Table begin
\begin{table}[ht]
    \begin{center}
    \caption{Accuracy after running Gaussian Naive Bayes on multi-source data after applying MDA (sigma=0.0001)}
    \begin{tabular}[c]{|c|c|c|c|c|c|}
        \hline
        Pair & 500 & 1000 & 2000 & Total \\
        \hline                             
        $SH, QF, BB \rightarrow AF$ & 0.703590992 & 0.723676202  & 0.723676202   & 0.69872185 \\ %
        \hline
    \end{tabular}
    \label{multisourcesigma00001}
   \end{center}
\end{table}
%==================Table end




%==================Table begin
\begin{table}[ht]
    \begin{center}
    \caption{Accuracy after running Gaussian Naive Bayes on multi-source data after applying MDA (sigma=0.01)}
    \begin{tabular}[c]{|c|c|c|c|c|c|}
        \hline
        Pair & 500 & 1000 & 2000 & Total \\
        \hline                             
        $SH, QF, BB \rightarrow AF$ & 0.703590992 & 0.723676202 & 0.723676202 & 0.69872185 \\ %
        \hline
    \end{tabular}
    \label{multisourcesigma001}
   \end{center}
\end{table}
%==================Table end



%==================Table begin
\begin{table}[ht]
    \begin{center}
    \caption{Accuracy after running Gaussian Naive Bayes on multi-source data after applying MDA (sigma=1)}
    \begin{tabular}[c]{|c|c|c|c|c|c|}
        \hline
        Pair & 500 & 1000 & 2000 & Total \\
        \hline                             
        $SH, QF, BB \rightarrow AF$ & 0.703590992 & 0.723676202 & 0.723676202 & 0.69872185 \\ %
        \hline
    \end{tabular}
    \label{multisourcesigma1}
   \end{center}
\end{table}
%==================Table end



%==================Table begin
\begin{table}[ht]
    \begin{center}
    \caption{Accuracy after running Gaussian Naive Bayes on multi-source data after applying MDA (sigma=10)}
    \begin{tabular}[c]{|c|c|c|c|c|c|}
        \hline
        Pair & 500 & 1000 & 2000 & Total \\
        \hline                             
        $SH, QF, BB \rightarrow AF$ & 0.702982349 & 0.721241631 & 0.717589775 & 0.687157638 \\ %
        \hline
    \end{tabular}
    \label{multisourcesigma10}
   \end{center}
\end{table}
%==================Table end



%==================Table begin
\begin{table}[ht]
    \begin{center}
    \caption{Accuracy after running Gaussian Naive Bayes on multi-source data after applying MDA (sigma=100)}
    \begin{tabular}[c]{|c|c|c|c|c|c|}
        \hline
        Pair & 500 & 1000 & 2000 & Total \\
        \hline                             
        $SH, QF, BB \rightarrow AF$ & 0.704808278 & 0.693852708 & 0.661594644 & 0.637857578 \\ %
        \hline
    \end{tabular}
    \label{multisourcesigma100}
   \end{center}
\end{table}
%==================Table end


We decide to further experiment with multi-source domain adaptation. Precisely, motivated by the improved results discussed in Chapter \ref{coralchapter}, we apply the same logic to the multi-source case: we first select features based on Variance Threshold, and then run CORAL. The transformed data is used in training of the Gaussian Naive Bayes classifier. The results are presented in Table \ref{multisourcenvt99coral}. We can see that the accuracy increases across all pairs, e.g. from $0.61$, when no domain adaptation is performed, to $0.81$, after applying feature selection and CORAL, for the pair $SH, QF \rightarrow BB$, when $500$ samples per class are used.


%==================Table begin
\begin{table}[ht]
    \begin{center}
    \caption{Accuracy after running Gaussian Naive Bayes on multi-source data after applying Variance Threshold (0.99) and CORAL}
    \begin{tabular}[c]{|c|c|c|c|c|c|}
        \hline
        Pair & 500 & 1000 & 2000 & Total \\
        \hline
                             
        $SH, QF \rightarrow BB$ & 0.811885714 & 	0.7952	& 0.783542857	& 0.795771429  \\ %
                            
        $SH, BB \rightarrow WT$ & 0.924119594 & 	0.943992949	& 0.94241173 & 	0.932474734   \\%
                             
        $SH, QF, BB \rightarrow OT$ &  0.868976026	& 0.862679013	& 0.820055275	& 0.840034068 \\%
                             
        $SH, QF, BB \rightarrow AF$ & 0.862623332	& 0.866762695	& 0.879185605	& 0.857628755 \\ %


        \hline
    \end{tabular}
    \label{multisourcenvt99coral}
   \end{center}
\end{table}
%==================Table end

In addition, we also confirm that applying CORAL contributes to the accuracy improvement by running an additional set of experiments. Precisely, we do not apply CORAL but we do apply feature selection (Variance Threshold). The results are presented in Table \ref{multisourcenvt99}. We can see that feature selection by itself improves performance, although the accuracy is still lower than when CORAL is applied.


%==================Table begin
\begin{table}[ht]
    \begin{center}
    \caption{Accuracy with Bernoulli Naive Bayes on multi-source data after applying Variance Threshold (0.99)}
    \begin{tabular}[c]{|c|c|c|c|c|c|}
        \hline
        Pair & 500 & 1000 & 2000 & Total \\
        \hline
                             
        $SH, QF \rightarrow BB$ & 0.727428571 &  0.7704  & 0.734285714 & 0.749714286 \\ %
                            
        $SH, BB \rightarrow WT$ &  0.92118289   &  0.928410509  & 0.926264574  & 0.924796731  {}\\%
                             
        $SH, QF, BB \rightarrow OT$ &  0.836763617  &  0.823445032  & 0.82925727   & 0.839428228 \\%
                             
        $SH, QF, BB \rightarrow AF$ & 0.740836072    & 0.741079455  & 0.748629961  & 0.73413774 \\ %


        \hline
    \end{tabular}
    \label{multisourcenvt99}
   \end{center}
\end{table}
%==================Table end

% +--------------------------------------------------------------------+
% | References
% +--------------------------------------------------------------------+

% +--------------------------------------------------------------------+
% | Included for Gather Purpose only.  Do NOT uncomment the next line.
%input "references.bib"
% | In order for the WinEDT editor to index references correctly, it
% | has to know where the "references.bib" file resides.  This
% | command will be ignored completely by LaTeX
% |
% | WinEDT can read file path names with either "\" or "/". LaTeX,
% | however,doesn't like "\", so it's easier to store a path name
% | using forward slashes "/".
% +--------------------------------------------------------------------+

\cleardoublepage
\phantomsection

% +--------------------------------------------------------------------+
% | This template uses the BibTeX program to format references.  The
% | lines below create a separate Bibliography section and add
% | an entry for "Bibliography" to the Table of Contents.  The actual
% | data for your references (author, title, journal, date, etc.) are
% | entered in the references.bib file.  See "Citations and Bibliography"
% | for details on to creating citations and formatting references.
% +--------------------------------------------------------------------+

\addcontentsline{toc}{chapter}{Bibliography}
\bibdata{references}
\bibliography{references}

% +--------------------------------------------------------------------+
% | The following commands add the appendices  To add or delete
% | appendices, add or remove the line
% |
% |     \input{appendixX.tex}
% |
% | where "X" is the letter designation of the appendix (A, B, C,
% | etc.) You should have one \input{appendixX.tex} line and a
% | corresponding file appendixX.tex for each appendix.
% |
% |If you do not have any appendices, comment out or delete the three
% |lines below.
% +--------------------------------------------------------------------+

%\appendix
%% +--------------------------------------------------------------------+
% | Appendix A Page (Optional)                                         
% +--------------------------------------------------------------------+

\cleardoublepage

\chapter{Title for This Appendix}

\label{Appendix:Key1}

Enter the content for Appendix A in the appendixA.tex file.  If you
do not have an Appendix A, see comments in the etdrtemplate.tex file
for instructions on how to remove this page.

%% +--------------------------------------------------------------------+
% | Appendix B Page (Optional)                                         
% +--------------------------------------------------------------------+

\cleardoublepage

\chapter{Title for This Appendix}
\label{Appendix:Key2}

Enter the content for Appendix B in the appendixB.tex file. If you
do not have an Appendix B, see comments in the etdrtemplate.tex file
for instructions on how to remove this page.


\end{document}

% +--------------------------------------------------------------------+
% | Template Revisions
% |
% | 9/14/06: Removed typos
% | 3/29/13: Removed hypernat package
% | 4/5/13: Changed to plain bib style
% | 5/17/13: added /cleardoublepage and /phantomsection to
% |          /bibliography to correct TOC page problem
% | 5/17/13: Fixed TOC problem with Dedication, Preface, etc.
% | 12/16/15: Added tocloft package to produce leader dots for all
% |           entries in the table of contents.
% |           Added geometry package to specify 1 inch margins.
% |           Removed unnecessary color specifications.
% |           Changed to \citep for citations.
% | 2/9/2016: Replaced \bibpunct with \setcitestyle.
% |           Changed to unsrtnat style
% |           Added natbib.pdf and Citations and Bibliography.pdf files
% |
% +--------------------------------------------------------------------+
